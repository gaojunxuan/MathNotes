%
% This is a borrowed LaTeX template file for lecture notes for CS267,
% Applications of Parallel Computing, UCBerkeley EECS Department.
%

\documentclass[twoside]{article}
\usepackage{titlesec}
\setlength{\oddsidemargin}{0.25 in}
\setlength{\evensidemargin}{-0.25 in}
\setlength{\topmargin}{-0.6 in}
\setlength{\textwidth}{6.5 in}
\setlength{\textheight}{8.5 in}
\setlength{\headsep}{0.75 in}
\setlength{\parindent}{0 in}
\setlength{\parskip}{0.1 in}


%
% ADD PACKAGES here:
%

\usepackage{amssymb}	% Already loads amsfonts
\usepackage{amsthm}
\usepackage{graphicx}
\usepackage{mathtools}	% Already loads amsmath
\usepackage{hyperref}
\usepackage{enumitem}
\usepackage{clrscode3e}  % for typesetting pseudocode
\usepackage{ulem}
\usepackage[usenames,dvipsnames]{xcolor}
\usepackage{soul}
\usepackage{cancel}

% Tikz and setup
\usepackage{tikz}
\usepackage{tikz-cd}
\usetikzlibrary{intersections, angles, quotes, calc, positioning}
\usetikzlibrary{arrows.meta}
\usepackage{pgfplots}
\pgfplotsset{compat=1.13}


\tikzset{
    force/.style={thick, {Circle[length=2pt]}-stealth, shorten <=-1pt}
}

%
% The following commands set up the lecnum (lecture number)
% counter and make various numbering schemes work relative
% to the lecture number.
%
\newcounter{lecnum}
\renewcommand{\thepage}{\thelecnum-\arabic{page}}
\renewcommand{\thesection}{\thelecnum.\arabic{section}}
\renewcommand{\theequation}{\thelecnum.\arabic{equation}}
\renewcommand{\thefigure}{\thelecnum.\arabic{figure}}
\renewcommand{\thetable}{\thelecnum.\arabic{table}}

%
% The following macro is used to generate the header.
%
\newcommand{\lecture}[5]{
   \pagestyle{myheadings}
   \thispagestyle{plain}
   \newpage
   \setcounter{lecnum}{#2}
   \setcounter{page}{1}
   \noindent
   \begin{center}
   \framebox{
      \vbox{\vspace{2mm}
    \hbox to 6.28in { {\bf #1
	\hfill} }
       \vspace{4mm}
       \hbox to 6.28in { {\Large \hfill Lecture #2: #3  \hfill} }
       \vspace{2mm}
       \hbox to 6.28in { {\it Lecturer: #4 \hfill Scribe: #5} }
      \vspace{2mm}}
   }
   \end{center}
   \markboth{Lecture #2: #3}{Lecture #2: #3}
   \vspace*{4mm}
}
\renewcommand{\cite}[1]{[#1]}
\def\beginrefs{\begin{list}%
        {[\arabic{equation}]}{\usecounter{equation}
         \setlength{\leftmargin}{2.0truecm}\setlength{\labelsep}{0.4truecm}%
         \setlength{\labelwidth}{1.6truecm}}}
\def\endrefs{\end{list}}
\def\bibentry#1{\item[\hbox{[#1]}]}

\newcommand{\fig}[3]{
			\vspace{#2}
			\begin{center}
			Figure \thelecnum.#1:~#3
			\end{center}
	}

% Colored theorem styles
\makeatother
\usepackage{thmtools}
\usepackage[framemethod=TikZ]{mdframed}
\mdfsetup{skipabove=1em,skipbelow=0.5em}

\declaretheoremstyle[
    headfont=\bfseries\sffamily\color{ForestGreen!70!black}, bodyfont=\normalfont,
    mdframed={
        linewidth=2pt,
        rightline=false, topline=false, bottomline=false,
        linecolor=ForestGreen, backgroundcolor=ForestGreen!5,
    },
    spaceabove=8pt
]{thmgreenbox}

\declaretheoremstyle[
    headfont=\bfseries\sffamily\color{NavyBlue!70!black}, bodyfont=\normalfont,
    mdframed={
        linewidth=2pt,
        rightline=false, topline=false, bottomline=false,
        linecolor=NavyBlue, backgroundcolor=NavyBlue!5,
    },
    spaceabove=8pt
]{thmbluebox}

\declaretheoremstyle[
    headfont=\bfseries\sffamily\color{NavyBlue!70!black}, bodyfont=\normalfont,
    mdframed={
        linewidth=2pt,
        rightline=false, topline=false, bottomline=false,
        linecolor=NavyBlue
    },
    spaceabove=8pt
]{thmblueline}

\declaretheoremstyle[
    headfont=\bfseries\sffamily\color{RawSienna!70!black}, bodyfont=\normalfont,
    mdframed={
        linewidth=2pt,
        rightline=false, topline=false, bottomline=false,
        linecolor=RawSienna, backgroundcolor=RawSienna!5,
    },
    spaceabove=8pt
]{thmredbox}

\declaretheoremstyle[
    headfont=\bfseries\sffamily\color{RawSienna!70!black}, bodyfont=\normalfont,
    numbered=no,
    mdframed={
        linewidth=2pt,
        rightline=false, topline=false, bottomline=false,
        linecolor=RawSienna, backgroundcolor=RawSienna!1,
    },
    qed=\qedsymbol,
    spaceabove=8pt
]{thmproofbox}

\declaretheoremstyle[
    headfont=\bfseries\sffamily\color{NavyBlue!70!black}, bodyfont=\normalfont,
    numbered=no,
    mdframed={
        linewidth=2pt,
        rightline=false, topline=false, bottomline=false,
        linecolor=NavyBlue, backgroundcolor=NavyBlue!1,
    },
    spaceabove=8pt
]{thmexplanationbox}

% Use these for theorems, lemmas, proofs, etc.
\theoremstyle{definition}
\declaretheorem[style=thmgreenbox, name=Definition, numberwithin=lecnum]{definition}
\declaretheorem[style=thmbluebox, numbered=no, name=Example]{example}
\declaretheorem[style=thmredbox, name=Proposition, numberwithin=lecnum]{proposition}
\declaretheorem[style=thmredbox, name=Theorem, numberwithin=lecnum]{theorem}
\declaretheorem[style=thmredbox, name=Lemma, sibling=theorem]{lemma}
\declaretheorem[style=thmredbox, name=Corollary, sibling=theorem]{corollary}
% \newtheorem{theorem}{Theorem}[lecnum]
% \newtheorem{lemma}[theorem]{Lemma}
% \newtheorem{claim}[theorem]{Claim}
% \newtheorem{corollary}[theorem]{Corollary}
% \newtheorem{definition}[theorem]{Definition}
\declaretheorem[style=thmblueline, numbered=no, name=Remark]{remark}
\declaretheorem[style=thmblueline, numbered=no, name=Conjecture]{conjecture}
\renewenvironment{proof}{{\bf \textit{Proof.}}}{\hfill\rule{2mm}{2mm}}
\makeatletter


% **** IF YOU WANT TO DEFINE ADDITIONAL MACROS FOR YOURSELF, PUT THEM HERE:

\renewcommand\Pr{\mathbb{P}}
\newcommand\Ex{\mathbb{E}}

\newcommand\N{\mathbb{N}}
\newcommand\Z{\mathbb{Z}}
\newcommand\Q{\mathbb{Q}}
\newcommand\R{\mathbb{R}}
\newcommand\C{\mathbb{C}}
\newcommand\F{\mathbb{F}}

\DeclarePairedDelimiter\ceil{\lceil}{\rceil}
\DeclarePairedDelimiter\floor{\lfloor}{\rfloor}
\DeclarePairedDelimiter\anglebrac{\langle}{\rangle}

\newcommand{\divides}{\mathrel{\mid}}
\newcommand{\notdivides}{\mathrel{\nmid}}

\usepackage{xr}
\externaldocument{unit4}

\newcommand\lcm{\mathrm{lcm}}

\begin{document}
\lecture{MATH453 Elementary Number Theory}{9}{Linear Congruence}{Bruce Berndt}{Kevin Gao}

\section{Linear Congruence}

We are all familiar with linear equations like
$$
ax + b = 0
$$
We are going to talk about a different kind of linear ``equation'' known as linear congruence. They are of the form
$$
ax \equiv b \mod m
$$
Let's consider some linear congruence. First, we look at $3x \equiv 1 \mod 6$. If we try everything from 0 to 5, it is easy to notice that this does not have a solution.

How about $2x \equiv 4 \mod 6$. We have one solution $\{2,8,14,\ldots,\}$ and another solution $\{5,11,17,\ldots\}$. In total, we have two solutions. Further, these two solution sets are incongruent because elements from one set is not congruent to those from the other set modulo $6$.

$2x \equiv 5 \mod 6$ does not have a solution. But $3x \equiv 1 \mod 5$ have the solution $\{2,7,12,\ldots,\}$.

The two examples that do not have a solution are $3x \equiv 1 \mod 6$ and $2x \equiv 5 \mod 6$. The ones that do have solutions are $2x \equiv 4 \mod 6$ and $3x \equiv 1 \mod 5$. From these examples, we make the observation that one thing in common among the linear congruences that do not have a solution is that the GCD of $a$ and $m$ does not divide $b$. In particular, we have that $\gcd(3,6) \notdivides 1$ and $\gcd(2,6) \notdivides 5$. On the other hand, $\gcd(2,6) \divides 4$ and $\gcd(3,5) \divides 1$.

We can generalize this into the following theorem:

\begin{theorem}
    Let $ax \equiv b \mod m$ be a linear congruence in one variable and let $d = \gcd(a,m)$. If $d \notdivides b$, then the linear congruence has no solution in $\Z$. If $d \divides b$, then the linear congruence has exactly $d$ incongruent solutions modulo $m$ in $\Z$.
\end{theorem}

The proof can be broken into two parts: 

(1) Showing that if $d \notdivides b$, then the linear congruence has no solution. This can be proved using the contrapositive of the original statement (namely, if the linear congruence has solution, then $d \divides b$); 

(2) Showing that if $d \divides b$, then the linear congruence has a solution, $x_0$. And given a solution $x_0$, we can construct infinitely many solutions of a given form and that among those infinitely many solutions, we have $d$ incongruent solutions. More specifically, it suffices to show the following:

\begin{addmargin}[1em]{0em}
    a. Show that a solution $x_0$ exists. \\
    b. Given a solution $x_0$, show that $ax \equiv b \mod m$ has infinitely many solutions in $\Z$ of a given form. \\
    c. Given a solution $x_0$, show that every solution has the form in (b). \\
    d. Show that there are $d$ incongruent solutions.
\end{addmargin}

\begin{proof}
    We begin the proof by proving Part (1) by its contrapositive.
    
    Assume that $ax \equiv b \mod m$ has a solution. By definition of congruence, $m \divides ax - b$. By definition of divisibility, $m \divides ax - b$ iff there exists some $y \in \Z$ such that $my = ax-b$. This, in turn, is true iff $ax - my = b$ has a solution. Since $d$ is the gcd of $a$ and $m$, $d \divides a$ and $d \divides m$. By Proposition 2.2, it follows that $d \divides ax - my$. Since $b = ax - my$ iff $ax \equiv b \mod m$ has a solution, $d \divides b$ if the original linear congruence has a solution.

    Now we proceed to prove Part (2)a. Assume $d \divides b$. Since $d$ is the gcd of $a$ and $m$, by Proposition \ref{prop:gcd-linear-combination}, there exists $r,s \in \Z$ such that
    $$
    d = ar + ms
    $$
    Further, $d \divides b$ implies $b = de$ for some $e \in \Z$. So, by substitution
    $$
    b = de = (ar + ms)e = a(re) + m(se)
    $$
    which clearly suggests a solution with $x = re$ and $y = -se$ that solves $ax - my = b$ (and thus solves $ax \equiv b \mod m$).

    For Part (2)b, let $x_0$ be an arbitrary solution for the linear congruence $ax \equiv b \mod m$. Let $n \in \Z$ and we consider
    $$
    x = x_0 + \left( \frac{m}{d} \right) n
    $$
    Since $d \divides m$, $\frac{m}{d}$ is an integer, so it follows that $x$ is also an integer. Furthermore, we observe that
    $$
    \begin{aligned}
        a\left( x_0 + \left( \frac{m}{d} \right) n \right) &= ax_0 + a\left( \frac{m}{d} \right) n \\
        &= ax_0 + \left( \frac{a}{d} \right) mn \\
        &\equiv ax_0 \mod m \\
        &\equiv b \mod m
    \end{aligned}
    $$
    Since for every solution $x$, $ax \equiv b \mod m$ but also $b \equiv a(x_0 + (\frac{m}{d})n) \mod m$, for all $n \in \Z$,
    $$
    x_0 + \left( \frac{m}{d} \right) n
    $$
    is also a solution to $ax \equiv b \mod m$.
\end{proof}

\end{document}