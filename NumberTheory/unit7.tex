%
% This is a borrowed LaTeX template file for lecture notes for CS267,
% Applications of Parallel Computing, UCBerkeley EECS Department.
%

\documentclass[twoside]{article}
\usepackage{titlesec}
\setlength{\oddsidemargin}{0.25 in}
\setlength{\evensidemargin}{-0.25 in}
\setlength{\topmargin}{-0.6 in}
\setlength{\textwidth}{6.5 in}
\setlength{\textheight}{8.5 in}
\setlength{\headsep}{0.75 in}
\setlength{\parindent}{0 in}
\setlength{\parskip}{0.1 in}


%
% ADD PACKAGES here:
%

\usepackage{amssymb}	% Already loads amsfonts
\usepackage{amsthm}
\usepackage{graphicx}
\usepackage{mathtools}	% Already loads amsmath
\usepackage{hyperref}
\usepackage{enumitem}
\usepackage{clrscode3e}  % for typesetting pseudocode
\usepackage{ulem}
\usepackage[usenames,dvipsnames]{xcolor}
\usepackage{soul}
\usepackage{cancel}

% Tikz and setup
\usepackage{tikz}
\usepackage{tikz-cd}
\usetikzlibrary{intersections, angles, quotes, calc, positioning}
\usetikzlibrary{arrows.meta}
\usepackage{pgfplots}
\pgfplotsset{compat=1.13}


\tikzset{
    force/.style={thick, {Circle[length=2pt]}-stealth, shorten <=-1pt}
}

%
% The following commands set up the lecnum (lecture number)
% counter and make various numbering schemes work relative
% to the lecture number.
%
\newcounter{lecnum}
\renewcommand{\thepage}{\thelecnum-\arabic{page}}
\renewcommand{\thesection}{\thelecnum.\arabic{section}}
\renewcommand{\theequation}{\thelecnum.\arabic{equation}}
\renewcommand{\thefigure}{\thelecnum.\arabic{figure}}
\renewcommand{\thetable}{\thelecnum.\arabic{table}}

%
% The following macro is used to generate the header.
%
\newcommand{\lecture}[5]{
   \pagestyle{myheadings}
   \thispagestyle{plain}
   \newpage
   \setcounter{lecnum}{#2}
   \setcounter{page}{1}
   \noindent
   \begin{center}
   \framebox{
      \vbox{\vspace{2mm}
    \hbox to 6.28in { {\bf #1
	\hfill} }
       \vspace{4mm}
       \hbox to 6.28in { {\Large \hfill Lecture #2: #3  \hfill} }
       \vspace{2mm}
       \hbox to 6.28in { {\it Lecturer: #4 \hfill Scribe: #5} }
      \vspace{2mm}}
   }
   \end{center}
   \markboth{Lecture #2: #3}{Lecture #2: #3}
   \vspace*{4mm}
}
\renewcommand{\cite}[1]{[#1]}
\def\beginrefs{\begin{list}%
        {[\arabic{equation}]}{\usecounter{equation}
         \setlength{\leftmargin}{2.0truecm}\setlength{\labelsep}{0.4truecm}%
         \setlength{\labelwidth}{1.6truecm}}}
\def\endrefs{\end{list}}
\def\bibentry#1{\item[\hbox{[#1]}]}

\newcommand{\fig}[3]{
			\vspace{#2}
			\begin{center}
			Figure \thelecnum.#1:~#3
			\end{center}
	}

% Colored theorem styles
\makeatother
\usepackage{thmtools}
\usepackage[framemethod=TikZ]{mdframed}
\mdfsetup{skipabove=1em,skipbelow=0.5em}

\declaretheoremstyle[
    headfont=\bfseries\sffamily\color{ForestGreen!70!black}, bodyfont=\normalfont,
    mdframed={
        linewidth=2pt,
        rightline=false, topline=false, bottomline=false,
        linecolor=ForestGreen, backgroundcolor=ForestGreen!5,
    },
    spaceabove=8pt
]{thmgreenbox}

\declaretheoremstyle[
    headfont=\bfseries\sffamily\color{NavyBlue!70!black}, bodyfont=\normalfont,
    mdframed={
        linewidth=2pt,
        rightline=false, topline=false, bottomline=false,
        linecolor=NavyBlue, backgroundcolor=NavyBlue!5,
    },
    spaceabove=8pt
]{thmbluebox}

\declaretheoremstyle[
    headfont=\bfseries\sffamily\color{NavyBlue!70!black}, bodyfont=\normalfont,
    mdframed={
        linewidth=2pt,
        rightline=false, topline=false, bottomline=false,
        linecolor=NavyBlue
    },
    spaceabove=8pt
]{thmblueline}

\declaretheoremstyle[
    headfont=\bfseries\sffamily\color{RawSienna!70!black}, bodyfont=\normalfont,
    mdframed={
        linewidth=2pt,
        rightline=false, topline=false, bottomline=false,
        linecolor=RawSienna, backgroundcolor=RawSienna!5,
    },
    spaceabove=8pt
]{thmredbox}

\declaretheoremstyle[
    headfont=\bfseries\sffamily\color{RawSienna!70!black}, bodyfont=\normalfont,
    numbered=no,
    mdframed={
        linewidth=2pt,
        rightline=false, topline=false, bottomline=false,
        linecolor=RawSienna, backgroundcolor=RawSienna!1,
    },
    qed=\qedsymbol,
    spaceabove=8pt
]{thmproofbox}

\declaretheoremstyle[
    headfont=\bfseries\sffamily\color{NavyBlue!70!black}, bodyfont=\normalfont,
    numbered=no,
    mdframed={
        linewidth=2pt,
        rightline=false, topline=false, bottomline=false,
        linecolor=NavyBlue, backgroundcolor=NavyBlue!1,
    },
    spaceabove=8pt
]{thmexplanationbox}

% Use these for theorems, lemmas, proofs, etc.
\theoremstyle{definition}
\declaretheorem[style=thmgreenbox, name=Definition, numberwithin=lecnum]{definition}
\declaretheorem[style=thmbluebox, numbered=no, name=Example]{example}
\declaretheorem[style=thmredbox, name=Proposition, numberwithin=lecnum]{proposition}
\declaretheorem[style=thmredbox, name=Theorem, numberwithin=lecnum]{theorem}
\declaretheorem[style=thmredbox, name=Lemma, sibling=theorem]{lemma}
\declaretheorem[style=thmredbox, name=Corollary, sibling=theorem]{corollary}
% \newtheorem{theorem}{Theorem}[lecnum]
% \newtheorem{lemma}[theorem]{Lemma}
% \newtheorem{claim}[theorem]{Claim}
% \newtheorem{corollary}[theorem]{Corollary}
% \newtheorem{definition}[theorem]{Definition}
\declaretheorem[style=thmblueline, numbered=no, name=Remark]{remark}
\declaretheorem[style=thmblueline, numbered=no, name=Conjecture]{conjecture}
\renewenvironment{proof}{{\bf \textit{Proof.}}}{\hfill\rule{2mm}{2mm}}
\makeatletter


% **** IF YOU WANT TO DEFINE ADDITIONAL MACROS FOR YOURSELF, PUT THEM HERE:

\renewcommand\Pr{\mathbb{P}}
\newcommand\Ex{\mathbb{E}}

\newcommand\N{\mathbb{N}}
\newcommand\Z{\mathbb{Z}}
\newcommand\Q{\mathbb{Q}}
\newcommand\R{\mathbb{R}}
\newcommand\C{\mathbb{C}}
\newcommand\F{\mathbb{F}}

\DeclarePairedDelimiter\ceil{\lceil}{\rceil}
\DeclarePairedDelimiter\floor{\lfloor}{\rfloor}
\DeclarePairedDelimiter\anglebrac{\langle}{\rangle}

\newcommand{\divides}{\mathrel{\mid}}
\newcommand{\notdivides}{\mathrel{\nmid}}

\newcommand\lcm{\mathrm{lcm}}

\begin{document}
\lecture{MATH453 Elementary Number Theory}{7}{Congruence}{Bruce Berndt}{Kevin Gao}

\section{Congruence}

\begin{definition}
    $a$ is congruent to $b$ modulo $m$ for $m \in \Z^+$ iff
    $$
    m \divides (a-b)
    $$
    and we write $a \equiv b \mod m$.
\end{definition}

For example, $3 \equiv 7 \mod 2$ because $3-7 = -4$ and $2 \divides -4$.

\begin{remark}
    Note that despite that congruence is denoted by $\equiv$, some properties of equality does not hold. Importantly, $ca \equiv cb \mod m$ DOES NOT imply $a \equiv b \mod m$. For a simple counterexample, consider $4 \equiv 6 \mod 2$ but $2 \not\equiv 3 \mod 2$.
\end{remark}

\subsection{Properties of Congruence Relation}

$$
\begin{aligned}
    &a \equiv a \mod m & \text{reflexive} \\
    &a \equiv b \mod m \iff b \equiv a \mod m & \text{symmetric} \\
    &a \equiv b \mod m \;\land\; b \equiv c \mod m \implies a \equiv c \mod m & \text{transitive}
\end{aligned}
$$

The reflexive and symmetric properties are obvious. We will provide a short proof for the transitive property.

\begin{proof}
    By definition of congruence, $a \equiv b \mod m$ means $m \divides (a-b)$. And $b \equiv c \mod m$ means $m \divides (b-c)$. It follows by property of divisibility that $m \divides (a-b+b-c)$. Then, $m \divides (a-c)$, which by definition means $a \equiv c \mod m$.
\end{proof}

Because of these three properties, we say that congruence defines an \textbf{equivalence relation}. Hence, equivalence relation of congruence divides integers into \textbf{equivalence classes}, known as the \textbf{congruence classes} or \textbf{residue classes}.

\subsection{Congruence Classes}

\begin{definition}[Congruence Classes]
    The congruence class of $a$ modulo $m$, denoted $[a]_m$, is the set of all integers that are congruent to $a$ modulo $m$
    $$
    \{ z \in \Z \mid m \divides (a-z) \}
    $$
\end{definition}

\begin{example}
    Let $m = 7$. Then,
    $$
    [0]_7 = \{\ldots,-14,-7,0,7,14,\ldots\}
    $$
    $$
    [1]_7 = \{\ldots,-13,-6,1,8,15,\ldots\}
    $$
    $$
    [2]_7 = \{\ldots,-12,-5,2,9,16,\ldots\}
    $$
    $$
    [3]_7 = \{\ldots,-11,-4,3,10,17,\ldots\}
    $$
\end{example}

\begin{definition}[Complete Residue System]
    A complete residue system modulo $m$ is a set $S$ of integers such that every $n \in \Z$ is congruent to one and only one member of $S$.
\end{definition}

\begin{example}
    $\{0,1,2,3,4,5,6\}$ is a complete residue system modulo $7$. \\ Although less obvious, $\{14,57,-12,1060,-24,-2,76\}$ is also a complete residue system modulo $7$.
\end{example}

\begin{proposition}
    $S = \{0,1,\ldots,m-1\}$ is a complete residue system modulo $m$.
\end{proposition}

\begin{proof}
    Let $a \in \Z$. Apply the division algorithm to $a$ with respect to $m$, so we have
    $$
    a = mq + r \qquad 0 \leq r \leq m-1
    $$
    By definition of divisibility, $m \divides (a-r)$, and by definition of congruence, $a \equiv r \mod m$. This shows that every integer is congruent to a member $r$ of $\{0,1,\ldots,m-1\}$.

    We also need to show that $a$ is congruent to only one member of $\{0,1,\ldots,m-1\}$. We proceed by contradiction. Assume $a \equiv r_1 \mod m$ and $a \equiv r_2 \mod m$ for some $r_1,r_2 \in \{0,1,\ldots,m-1\}$. By transitivity, $r_1 \equiv r_2 \mod m$, which by definition means $m \divides (r_1 - r_2)$. Since both $r_1$ and $r_2$ are between 0 and $m-1$, $0 \leq r_1-r_2 \leq m-1$. Then, $0 \leq r_1 - r_2 \leq m - 1$ and $m \divides (r_1-r_2)$ imply that $r_1-r_2 = 0$ because otherwise $m$ cannot divide any non-zero integers less than itself. This shows that $r_1 = r_2$ and thus uniqueness.
\end{proof}

\begin{proposition}
    Let $a,b,c,d \in \Z$. If $a \equiv b \mod m$ and $c \equiv d \mod m$, then
    \begin{equation} \label{eq:congruence-prop-1}
        a + c \equiv b + d \mod m
    \end{equation}
    \begin{equation} \label{eq:congruence-prop-2}
        ac \equiv bd \mod m
    \end{equation}
\end{proposition}

\begin{proof} of Equation (\ref{eq:congruence-prop-1})

    By definition of congruence, $m \divides (a-b)$ and $m \divides (c-d)$. By property of divisibility, $m \divides (a-b+c-d)$. This is equivalence to $m \divides [(a+c)-(b+d)]$, which by definition means $a+c \equiv b+d \mod m$.
\end{proof}

\begin{proof}
    of Equation (\ref{eq:congruence-prop-2})

    By definition, $m \divides (a-b)$ and $m \divides (c-d)$. Trivially, it follows that $m \divides c(a-b)$. Similarly, $m \divides b(c-d)$. By property of divisibility, $m \divides (ca-cb+bc-bd)$ so $m \divides (ac-bd)$. This by definition means $ac \equiv bd \mod m$.
\end{proof}

\end{document}