%
% This is a borrowed LaTeX template file for lecture notes for CS267,
% Applications of Parallel Computing, UCBerkeley EECS Department.
%

\documentclass[twoside]{article}
\usepackage{titlesec}
\setlength{\oddsidemargin}{0.25 in}
\setlength{\evensidemargin}{-0.25 in}
\setlength{\topmargin}{-0.6 in}
\setlength{\textwidth}{6.5 in}
\setlength{\textheight}{8.5 in}
\setlength{\headsep}{0.75 in}
\setlength{\parindent}{0 in}
\setlength{\parskip}{0.1 in}


%
% ADD PACKAGES here:
%

\usepackage{amssymb}	% Already loads amsfonts
\usepackage{amsthm}
\usepackage{graphicx}
\usepackage{mathtools}	% Already loads amsmath
\usepackage{hyperref}
\usepackage{enumitem}
\usepackage{clrscode3e}  % for typesetting pseudocode
\usepackage{ulem}
\usepackage[usenames,dvipsnames]{xcolor}
\usepackage{soul}
\usepackage{cancel}

% Tikz and setup
\usepackage{tikz}
\usepackage{tikz-cd}
\usetikzlibrary{intersections, angles, quotes, calc, positioning}
\usetikzlibrary{arrows.meta}
\usepackage{pgfplots}
\pgfplotsset{compat=1.13}


\tikzset{
    force/.style={thick, {Circle[length=2pt]}-stealth, shorten <=-1pt}
}

%
% The following commands set up the lecnum (lecture number)
% counter and make various numbering schemes work relative
% to the lecture number.
%
\newcounter{lecnum}
\renewcommand{\thepage}{\thelecnum-\arabic{page}}
\renewcommand{\thesection}{\thelecnum.\arabic{section}}
\renewcommand{\theequation}{\thelecnum.\arabic{equation}}
\renewcommand{\thefigure}{\thelecnum.\arabic{figure}}
\renewcommand{\thetable}{\thelecnum.\arabic{table}}

%
% The following macro is used to generate the header.
%
\newcommand{\lecture}[5]{
   \pagestyle{myheadings}
   \thispagestyle{plain}
   \newpage
   \setcounter{lecnum}{#2}
   \setcounter{page}{1}
   \noindent
   \begin{center}
   \framebox{
      \vbox{\vspace{2mm}
    \hbox to 6.28in { {\bf #1
	\hfill} }
       \vspace{4mm}
       \hbox to 6.28in { {\Large \hfill Lecture #2: #3  \hfill} }
       \vspace{2mm}
       \hbox to 6.28in { {\it Lecturer: #4 \hfill Scribe: #5} }
      \vspace{2mm}}
   }
   \end{center}
   \markboth{Lecture #2: #3}{Lecture #2: #3}
   \vspace*{4mm}
}
\renewcommand{\cite}[1]{[#1]}
\def\beginrefs{\begin{list}%
        {[\arabic{equation}]}{\usecounter{equation}
         \setlength{\leftmargin}{2.0truecm}\setlength{\labelsep}{0.4truecm}%
         \setlength{\labelwidth}{1.6truecm}}}
\def\endrefs{\end{list}}
\def\bibentry#1{\item[\hbox{[#1]}]}

\newcommand{\fig}[3]{
			\vspace{#2}
			\begin{center}
			Figure \thelecnum.#1:~#3
			\end{center}
	}

% Colored theorem styles
\makeatother
\usepackage{thmtools}
\usepackage[framemethod=TikZ]{mdframed}
\mdfsetup{skipabove=1em,skipbelow=0.5em}

\declaretheoremstyle[
    headfont=\bfseries\sffamily\color{ForestGreen!70!black}, bodyfont=\normalfont,
    mdframed={
        linewidth=2pt,
        rightline=false, topline=false, bottomline=false,
        linecolor=ForestGreen, backgroundcolor=ForestGreen!5,
    },
    spaceabove=8pt
]{thmgreenbox}

\declaretheoremstyle[
    headfont=\bfseries\sffamily\color{NavyBlue!70!black}, bodyfont=\normalfont,
    mdframed={
        linewidth=2pt,
        rightline=false, topline=false, bottomline=false,
        linecolor=NavyBlue, backgroundcolor=NavyBlue!5,
    },
    spaceabove=8pt
]{thmbluebox}

\declaretheoremstyle[
    headfont=\bfseries\sffamily\color{NavyBlue!70!black}, bodyfont=\normalfont,
    mdframed={
        linewidth=2pt,
        rightline=false, topline=false, bottomline=false,
        linecolor=NavyBlue
    },
    spaceabove=8pt
]{thmblueline}

\declaretheoremstyle[
    headfont=\bfseries\sffamily\color{RawSienna!70!black}, bodyfont=\normalfont,
    mdframed={
        linewidth=2pt,
        rightline=false, topline=false, bottomline=false,
        linecolor=RawSienna, backgroundcolor=RawSienna!5,
    },
    spaceabove=8pt
]{thmredbox}

\declaretheoremstyle[
    headfont=\bfseries\sffamily\color{RawSienna!70!black}, bodyfont=\normalfont,
    numbered=no,
    mdframed={
        linewidth=2pt,
        rightline=false, topline=false, bottomline=false,
        linecolor=RawSienna, backgroundcolor=RawSienna!1,
    },
    qed=\qedsymbol,
    spaceabove=8pt
]{thmproofbox}

\declaretheoremstyle[
    headfont=\bfseries\sffamily\color{NavyBlue!70!black}, bodyfont=\normalfont,
    numbered=no,
    mdframed={
        linewidth=2pt,
        rightline=false, topline=false, bottomline=false,
        linecolor=NavyBlue, backgroundcolor=NavyBlue!1,
    },
    spaceabove=8pt
]{thmexplanationbox}

% Use these for theorems, lemmas, proofs, etc.
\theoremstyle{definition}
\declaretheorem[style=thmgreenbox, name=Definition, numberwithin=lecnum]{definition}
\declaretheorem[style=thmbluebox, numbered=no, name=Example]{example}
\declaretheorem[style=thmredbox, name=Proposition, numberwithin=lecnum]{proposition}
\declaretheorem[style=thmredbox, name=Theorem, numberwithin=lecnum]{theorem}
\declaretheorem[style=thmredbox, name=Lemma, sibling=theorem]{lemma}
\declaretheorem[style=thmredbox, name=Corollary, sibling=theorem]{corollary}
% \newtheorem{theorem}{Theorem}[lecnum]
% \newtheorem{lemma}[theorem]{Lemma}
% \newtheorem{claim}[theorem]{Claim}
% \newtheorem{corollary}[theorem]{Corollary}
% \newtheorem{definition}[theorem]{Definition}
\declaretheorem[style=thmblueline, numbered=no, name=Remark]{remark}
\declaretheorem[style=thmblueline, numbered=no, name=Conjecture]{conjecture}
\renewenvironment{proof}{{\bf \textit{Proof.}}}{\hfill\rule{2mm}{2mm}}
\makeatletter


% **** IF YOU WANT TO DEFINE ADDITIONAL MACROS FOR YOURSELF, PUT THEM HERE:

\renewcommand\Pr{\mathbb{P}}
\newcommand\Ex{\mathbb{E}}

\newcommand\N{\mathbb{N}}
\newcommand\Z{\mathbb{Z}}
\newcommand\Q{\mathbb{Q}}
\newcommand\R{\mathbb{R}}
\newcommand\C{\mathbb{C}}
\newcommand\F{\mathbb{F}}

\DeclarePairedDelimiter\ceil{\lceil}{\rceil}
\DeclarePairedDelimiter\floor{\lfloor}{\rfloor}
\DeclarePairedDelimiter\anglebrac{\langle}{\rangle}

\newcommand{\divides}{\mathrel{\mid}}
\newcommand{\notdivides}{\mathrel{\nmid}}

\begin{document}
\lecture{MATH453 Elementary Number Theory}{5}{Prime Frequency and Factorization}{Bruce Berndt}{Kevin Gao}

\section{Distribution of Primes}

Recall that from calculus, we know that
$
\sum_{n=1}^{\infty} \frac{1}{n} = \infty
$
but
$
\sum_{n=1}^{\infty} \frac{1}{n^2} = \frac{\pi^2}{6}.
$
Now, one might be interested in knowing how the series
$$
\sum_{\text{$p$ prime}} \frac{1}{p}
$$
behaves.

\begin{theorem}
    For every $y \geq 2$,
    $$
    \sum_{\substack{p \leq y \\ \text{$p$ prime}}} \frac{1}{p} \geq \log \log y - 1
    $$
\end{theorem}

\begin{proof}
    Let $\mathcal{N}$ be the subset of $\Z^+$ whose prime factorizations contain only primes $\leq y$. Consider $\sum_{n=1}^{\floor{y}} 1/n$, which is an upper bound on the sum that we are interested in.
    \begin{equation}
    \sum_{n=1}^{\floor{y}} \frac{1}{n} \geq \int_{1}^{\floor{y} + 1} \frac{dx}{x} = \log (\floor{y}+1) \geq \log (y)
    \end{equation}
    Now, consider the product of the geometric series over all primes $p \leq y$.
    \begin{equation}
        \prod_{\substack{p \leq y \\ \text{$p$ prime}}} \left( \sum_{i=0}^\infty \frac{1}{p^i} \right) = \prod_{\substack{p \leq y \\ \text{$p$ prime}}} \frac{1}{1 - 1/p} = \sum_{n \in \mathcal{N}} \frac{1}{n} \geq \sum_{n=1}^{\floor{y}} \frac{1}{n} \geq \int_{1}^{\floor{y} + 1} \frac{dx}{x} \geq \log y
    \end{equation}
    \textit{Claim}. For $0 \leq v \leq \frac{1}{2}$, $e^{v+v^2} \geq \frac{1}{1-v}$. Let $v = 1/p \leq 1/2$.

    Then, from the claim,
    \begin{equation}
    \prod_{\substack{p \leq y \\ \text{$p$ prime}}} e^{\frac{1}{p} + \frac{1}{p^2}} \geq \prod_{\substack{p \leq y \\ \text{$p$ prime}}} \frac{1}{1-1/p} \geq \log y
    \end{equation}
    Take the logarithm of both sides,
    \begin{equation}
        \log \prod_{\substack{p \leq y \\ \text{$p$ prime}}} e^{\frac{1}{p} + \frac{1}{p^2}} \leq \sum_{\substack{p \leq y \\ \text{$p$ prime}}} \log e^{\frac{1}{p} + \frac{1}{p^2}} = \sum_{\substack{p \leq y \\ \text{$p$ prime}}} \left( \frac{1}{p} + \frac{1}{p^2} \right) \geq \log \log y 
    \end{equation}
    Further, we observe that
    \begin{equation}
        \sum_{\substack{p \leq y \\ \text{$p$ prime}}} \frac{1}{p^2} \leq \sum_{n=2}^{\infty} \frac{1}{n^2} = \frac{\pi^2}{6} < 1
    \end{equation}
    So it follows that
    \begin{equation}
        \sum_{\substack{p \leq y \\ \text{$p$ prime}}} \frac{1}{p} > \log \log y - \sum_{\substack{p \leq y \\ \text{$p$ prime}}} \frac{1}{p^2} > \log \log y - 1
    \end{equation}
\end{proof}

We must also prove the claim we have used in the proof of the theorem.

\begin{lemma}
    For $0 \leq v \leq 1/2$,
    $$
    e^{v+v^2} \geq \frac{1}{1-v}
    $$
\end{lemma}
\begin{proof}
    Let $f(v) = (1-v) e^{v + v^2}$. $f(0)= 1$. We see that the first derivative is non-negative and thus $f$ is non-decreasing on $[0,\frac{1}{2}]$.
    $$
    f'(v) = -e^{v+v^2} + (1-v)(1+2v)e^{v+v^2} = v(1-2v)e^{v+v^2}.
    $$
    Then, since $f(v) = (1-v) e^{v+v^2}$ is non-decreasing on $[0,1/2]$, we have
    $$
    f(v) = (1-v) e^{v+v^2} \geq f(0) = 1 \qquad \forall v \in [0,1/2]
    $$
    This implies that $e^{v+v^2} \geq \frac{1}{1-v}$ for $0 \leq v \leq 1/2$.
\end{proof}

\section{Fundamental Theorem of Arithmetic}

Now we introduce a few lemmas in preparation for the Fundamental Theorem of Arithmetic.

\begin{lemma} \label{lem:unique-factor-lem}
    Let $p$ be prime such that $p \divides ab$. Then, $p \divides a$ or $p \divides b$.
\end{lemma}

\begin{proof}
    Assume that $p \divides ab$. If $p \divides a$, then we are done, so assume that $p \notdivides a$. Then, $a$ and $p$ are coprime, so $\gcd(p,a) = 1$. There exists some $m,n \in \Z$ such that $ma + np = 1$ by Prop 4.2.

    $b = 1 \cdot b$ so $b = mab + npb$. By assumption, $p \divides ab$. Then, there exists $c \in \Z$ such that $ab = pc$. It follows that
    $$
    b = mpc + npb = p(mc+nb)
    $$
    which, by definition of divisiblity, $p \divides b$.
\end{proof}

\begin{corollary} \label{cor:fta-cor}
    Let $p$ be prime, $a_1,\ldots,a_n \in \Z$ for $n \geq 2$. If $p \divides a_1a_2\ldots a_n$, then $p \divides a_j$ for at least one $j \in \{1,2,\ldots,n\}$.
\end{corollary}

\begin{proof}
    By Lemma \ref{lem:unique-factor-lem}, $p \divides a_1\ldots a_{n-1}$ or $p \divides a_n$. Prove by induction on $n \geq 2$.
\end{proof}

\begin{theorem}[Fundamental Theorem of Arithmetic]
    Every integer $a > 1$ can be represented \textbf{uniquely} as a product of primes
    $$
    a = p_1^{a_1} p_2^{a_2} \cdots p_n^{a_n}
    $$
    where $p_i \neq p_j$ if $i \neq j$ for positive integers $a_i$.
\end{theorem}

Now, we are ready to prove the \textit{\textbf{Fundamental Theorem of Arithmetic}}. It states the factorizability of any positive integers so the theorem is sometimes called the unique factorization theorem.

\begin{proof}
    By contradiction.

    Assume that there exists some integer without a prime factorization. Take $c$ to be the smallest of such counterexamples. Then, $c$ must be composite (otherwise, $c$ itself would be a unique prime factorization of $c$). Then, $c = ab$ for some $a,b > 1$ and $a,b < c$. Since $c$ is the smallest counterexample, $a$ and $b$, which are smaller than $c$, can be represented as products of primes. Therefore, $c$ indeed has a prime factorization that is the product of the prime factorizations of $a$ and $b$. This is a contradiction, so $c$ \textbf{has a prime factorization}.

    It remains to be shown that the factorization of $c$ is unique. Suppose for contradiction that $c$ has two prime factorizations. That is
    $$
    c = p_1^{a_1} p_2^{a_2} \cdots p_m^{a_m} = q_1^{b_1} q_2^{b_2} \cdots q_n^{b_n}
    $$
    where $p_1 < p_{i+1}$ and $q_{j} < q_{j+1}$ for all $i \in \{1,\ldots,m-1\}$ and $j \in \{1,\ldots,n-1\}$.

    It suffices to show that $p_j = q_j$, $a_j = b_j$ for all $j$, $m = n$.
    
    Fix arbitrary $p_i$. By Corollary \ref{cor:fta-cor}, $p_i \divides q_j$ for some $j$. Since $p_i$ and $q_j$ are prime, it follows that $p_i = q_j$ because otherwise it would be a contradiction. Similarly, fix $q_j$, and by the same argument, $q_j \divides p_i$ for some $i$ so $p_i = q_j$. Thus, $p_j = q_j$ for all $j$. This also implies that $m = n$.

    Finally, we show that the exponents are also equal. Suppose for contradiction that there exists some $j$ such that $a_j \neq b_j$. Without loss of generality, assume $a_j < b_j$. Since
    $$
    p_j^{b_j} \divides c = p_1^{a_1} p_2^{a_2} \cdots p_n^{a_n}
    $$
    so $p_1^{a_1} p_2^{a_2} \cdots p_n^{a_n} = k p_j^{b_j}$ for some $k \in \Z$. It follows that by dividing both sides by $p_j^{a_j}$,
    $$
    p_1^{a_1} p_2^{a_2} \cdots p_{j-1}^{a_{j-1}} p_{j+1}^{a_{j+1}} \cdots p_n^{a_n} = k p_{j}^{b_j - a_j}
    $$
    Since $b_j - a_j > 0$, $p_j \divides p_1^{a_1} p_2^{a_2} \cdots p_{j-1}^{a_{j-1}} p_{j+1}^{a_{j+1}} \cdots p_n^{a_n}$. By Corollary \ref{cor:fta-cor}, $p_j \divides p_i$ for some $i \neq j$. But this is not possible because for all $i \in \{1\ldots n\} \setminus \{j\}$, $p_i$ is prime and $p_j$ is \textbf{not a factor} of $p_1^{a_1} p_2^{a_2} \cdots p_{j-1}^{a_{j-1}} p_{j+1}^{a_{j+1}} \cdots p_n^{a_n}$. Hence, $p_i \divides p_j$ and $p_i \neq p_j$, which is a contradiction because a prime cannot divide another prime.

    Therefore, the prime \textbf{factorization is unique}.
\end{proof}

\end{document}