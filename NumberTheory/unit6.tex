%
% This is a borrowed LaTeX template file for lecture notes for CS267,
% Applications of Parallel Computing, UCBerkeley EECS Department.
%

\documentclass[twoside]{article}
\usepackage{titlesec}
\setlength{\oddsidemargin}{0.25 in}
\setlength{\evensidemargin}{-0.25 in}
\setlength{\topmargin}{-0.6 in}
\setlength{\textwidth}{6.5 in}
\setlength{\textheight}{8.5 in}
\setlength{\headsep}{0.75 in}
\setlength{\parindent}{0 in}
\setlength{\parskip}{0.1 in}


%
% ADD PACKAGES here:
%

\usepackage{amssymb}	% Already loads amsfonts
\usepackage{amsthm}
\usepackage{graphicx}
\usepackage{mathtools}	% Already loads amsmath
\usepackage{hyperref}
\usepackage{enumitem}
\usepackage{clrscode3e}  % for typesetting pseudocode
\usepackage{ulem}
\usepackage[usenames,dvipsnames]{xcolor}
\usepackage{soul}
\usepackage{cancel}

% Tikz and setup
\usepackage{tikz}
\usepackage{tikz-cd}
\usetikzlibrary{intersections, angles, quotes, calc, positioning}
\usetikzlibrary{arrows.meta}
\usepackage{pgfplots}
\pgfplotsset{compat=1.13}


\tikzset{
    force/.style={thick, {Circle[length=2pt]}-stealth, shorten <=-1pt}
}

%
% The following commands set up the lecnum (lecture number)
% counter and make various numbering schemes work relative
% to the lecture number.
%
\newcounter{lecnum}
\renewcommand{\thepage}{\thelecnum-\arabic{page}}
\renewcommand{\thesection}{\thelecnum.\arabic{section}}
\renewcommand{\theequation}{\thelecnum.\arabic{equation}}
\renewcommand{\thefigure}{\thelecnum.\arabic{figure}}
\renewcommand{\thetable}{\thelecnum.\arabic{table}}

%
% The following macro is used to generate the header.
%
\newcommand{\lecture}[5]{
   \pagestyle{myheadings}
   \thispagestyle{plain}
   \newpage
   \setcounter{lecnum}{#2}
   \setcounter{page}{1}
   \noindent
   \begin{center}
   \framebox{
      \vbox{\vspace{2mm}
    \hbox to 6.28in { {\bf #1
	\hfill} }
       \vspace{4mm}
       \hbox to 6.28in { {\Large \hfill Lecture #2: #3  \hfill} }
       \vspace{2mm}
       \hbox to 6.28in { {\it Lecturer: #4 \hfill Scribe: #5} }
      \vspace{2mm}}
   }
   \end{center}
   \markboth{Lecture #2: #3}{Lecture #2: #3}
   \vspace*{4mm}
}
\renewcommand{\cite}[1]{[#1]}
\def\beginrefs{\begin{list}%
        {[\arabic{equation}]}{\usecounter{equation}
         \setlength{\leftmargin}{2.0truecm}\setlength{\labelsep}{0.4truecm}%
         \setlength{\labelwidth}{1.6truecm}}}
\def\endrefs{\end{list}}
\def\bibentry#1{\item[\hbox{[#1]}]}

\newcommand{\fig}[3]{
			\vspace{#2}
			\begin{center}
			Figure \thelecnum.#1:~#3
			\end{center}
	}

% Colored theorem styles
\makeatother
\usepackage{thmtools}
\usepackage[framemethod=TikZ]{mdframed}
\mdfsetup{skipabove=1em,skipbelow=0.5em}

\declaretheoremstyle[
    headfont=\bfseries\sffamily\color{ForestGreen!70!black}, bodyfont=\normalfont,
    mdframed={
        linewidth=2pt,
        rightline=false, topline=false, bottomline=false,
        linecolor=ForestGreen, backgroundcolor=ForestGreen!5,
    },
    spaceabove=8pt
]{thmgreenbox}

\declaretheoremstyle[
    headfont=\bfseries\sffamily\color{NavyBlue!70!black}, bodyfont=\normalfont,
    mdframed={
        linewidth=2pt,
        rightline=false, topline=false, bottomline=false,
        linecolor=NavyBlue, backgroundcolor=NavyBlue!5,
    },
    spaceabove=8pt
]{thmbluebox}

\declaretheoremstyle[
    headfont=\bfseries\sffamily\color{NavyBlue!70!black}, bodyfont=\normalfont,
    mdframed={
        linewidth=2pt,
        rightline=false, topline=false, bottomline=false,
        linecolor=NavyBlue
    },
    spaceabove=8pt
]{thmblueline}

\declaretheoremstyle[
    headfont=\bfseries\sffamily\color{RawSienna!70!black}, bodyfont=\normalfont,
    mdframed={
        linewidth=2pt,
        rightline=false, topline=false, bottomline=false,
        linecolor=RawSienna, backgroundcolor=RawSienna!5,
    },
    spaceabove=8pt
]{thmredbox}

\declaretheoremstyle[
    headfont=\bfseries\sffamily\color{RawSienna!70!black}, bodyfont=\normalfont,
    numbered=no,
    mdframed={
        linewidth=2pt,
        rightline=false, topline=false, bottomline=false,
        linecolor=RawSienna, backgroundcolor=RawSienna!1,
    },
    qed=\qedsymbol,
    spaceabove=8pt
]{thmproofbox}

\declaretheoremstyle[
    headfont=\bfseries\sffamily\color{NavyBlue!70!black}, bodyfont=\normalfont,
    numbered=no,
    mdframed={
        linewidth=2pt,
        rightline=false, topline=false, bottomline=false,
        linecolor=NavyBlue, backgroundcolor=NavyBlue!1,
    },
    spaceabove=8pt
]{thmexplanationbox}

% Use these for theorems, lemmas, proofs, etc.
\theoremstyle{definition}
\declaretheorem[style=thmgreenbox, name=Definition, numberwithin=lecnum]{definition}
\declaretheorem[style=thmbluebox, numbered=no, name=Example]{example}
\declaretheorem[style=thmredbox, name=Proposition, numberwithin=lecnum]{proposition}
\declaretheorem[style=thmredbox, name=Theorem, numberwithin=lecnum]{theorem}
\declaretheorem[style=thmredbox, name=Lemma, sibling=theorem]{lemma}
\declaretheorem[style=thmredbox, name=Corollary, sibling=theorem]{corollary}
% \newtheorem{theorem}{Theorem}[lecnum]
% \newtheorem{lemma}[theorem]{Lemma}
% \newtheorem{claim}[theorem]{Claim}
% \newtheorem{corollary}[theorem]{Corollary}
% \newtheorem{definition}[theorem]{Definition}
\declaretheorem[style=thmblueline, numbered=no, name=Remark]{remark}
\declaretheorem[style=thmblueline, numbered=no, name=Conjecture]{conjecture}
\renewenvironment{proof}{{\bf \textit{Proof.}}}{\hfill\rule{2mm}{2mm}}
\makeatletter


% **** IF YOU WANT TO DEFINE ADDITIONAL MACROS FOR YOURSELF, PUT THEM HERE:

\renewcommand\Pr{\mathbb{P}}
\newcommand\Ex{\mathbb{E}}

\newcommand\N{\mathbb{N}}
\newcommand\Z{\mathbb{Z}}
\newcommand\Q{\mathbb{Q}}
\newcommand\R{\mathbb{R}}
\newcommand\C{\mathbb{C}}
\newcommand\F{\mathbb{F}}

\DeclarePairedDelimiter\ceil{\lceil}{\rceil}
\DeclarePairedDelimiter\floor{\lfloor}{\rfloor}
\DeclarePairedDelimiter\anglebrac{\langle}{\rangle}

\newcommand{\divides}{\mathrel{\mid}}
\newcommand{\notdivides}{\mathrel{\nmid}}

\newcommand\lcm{\mathrm{lcm}}

\begin{document}
\lecture{MATH453 Elementary Number Theory}{6}{LCM and Dirichlet's Theorem}{Bruce Berndt}{Kevin Gao}

\section{Least Common Multiple}

\begin{definition}[Least Common Multiple]
    Let $a,b \neq 0$. The \textit{\textbf{least common multiple}} of $a$ and $b$, denoted by $[a,b]$ or $\lcm(a,b)$, is the least $m > 0$ such that $a \divides m$ and $b \divides m$.
\end{definition}

\begin{example}
    What is $\lcm(2^3 3^2 7^5,\, 2 \cdot 3^5 7 \cdot 11^2)$? \\
    We take the least common multiple of each factor, so we have $2^3 3^5 7^5 11^2$ 
\end{example}

If $\gcd(a,b) = 1$, $\lcm(a,b) = ab$. In general, we have the following theorem

\begin{theorem}
    $
    \gcd(a,b) \cdot \lcm(a,b) = ab
    $
\end{theorem}

\begin{proof}
    Assume $a,b > 1$. If either $a$ or $b$ is equal to 1, then the proof is trivial. Let
    $$
    a = p_1^{a_1} p_2^{a_2} \cdots p_n^{a_n} \qquad b = p_1^{b_1} p_2^{b_2} \cdots p_n^{b_n}
    $$
    so that $p_1,\ldots,p_n$ are the primes in common to $a$ and $b$. Since $\gcd(a,b)$ is the greatest common \textbf{divisor},
    $$
    \gcd(a,b) = p_1^{\min(a_1,b_1)} p_2^{\min(a_2,b_2)} \cdots p_n^{\min(a_n,b_n)}
    $$
    Also,
    $$
    \lcm(a,b) = p_1^{\max(a_1,b_1)} p_2^{\max(a_2,b_2)} \cdots p_n^{\max(a_n,b_n)}
    $$
    Multiply the two together and we get
    $$
    \gcd(a,b) \cdot \lcm(a,b) = p_1^{a_1+b_1} \cdots p_n^{a_n + b_n} = (p_1^{a_1} \cdots p_n^{a_n}) \cdot (p_1^{b_1} \cdots p_n^{b_n}) = ab
    $$
\end{proof}

In practice, we can use theorem to find the least common multiple once we find the greatest common divisor using the Euclidean algorithm.

\section{Arithmetic Progression and Dirichlet's Theorem}

Arithmetic progression is another name for the arithmetic sequence, a sequence of integers in which the difference between two consecutive numbers is constant. In general, the $n$th term in an arithmetic sequence/progression is given by
$$
a_n = a_1 + (n-1) d
$$
How many primes are there in an infinite arithmetic progression? The theorem of Dirichlet tells us that indeed there are infinitely many primes in an infinite arithmetic progression.

\begin{theorem}[Dirichlet's Theorem on Primes in Arithmetic Progression]
    If $\gcd(a,b)=1$ ($a$ and $b$ are \textbf{relatively prime}), then the set
    $$
    \{ an + b \mid n \in \Z,\, n \geq 0 \}
    $$
    has \textbf{infinitely} many primes.
\end{theorem}

The proof of Dirichlet's theorem is beyond the scope of this course but will be covered in a course on analytic number theory. We can, however, prove some special cases of Dirichlet's theorem.

\subsection{Special Cases of Dirichlet's Theorem}
\begin{lemma} \label{lem:dirichlet-special-case-lem}
    Let $a,b \in \Z^+$. Suppose $a,b \in \{4n+1 \mid n \in \Z,\, n \geq 0\}$. Then, $ab \in \{4n+1 \mid n \in \Z,\, n \geq 0\}$.
\end{lemma}

\begin{proof}
    Let $a = 4n_1 + 1$ and $b = 4n_2 + 1$. Then,
    $$
    ab = (4n_1 + 1) (4n_2 + 1) = 16n_1n_2 + 4(n_1+n_2) + 1
    $$
    which can be factored as $4(4n_1n_2 + n_1 + n_2) + 1$. Take $n = (4n_1n_2 + n_1 + n_2)$ which is clearly a non-negative integer. Then, $n = ab \in \{4n+1 \mid n \in \Z,\, n \geq 0\}$.
\end{proof}

Using this simple fact, we can show that there are infinitely many primes of the form $4n + 1$.

\begin{proposition}
    There exist infinitely many primes in $\{ 4n+3 \mid n \in \Z,\, n \geq 0\}$.
\end{proposition}

\begin{proof}
    By contradiction.

    Suppose for contradiction that there exist only finitely many primes in $\{ 4n+3 \mid n \in \Z,\, n \geq 0\}$. Say there exist only $r+1$ such primes. Clearly, $p_0 = 3$, and we have $p_0,p_1,\ldots,p_r$ from the set that are primes.

    Take $N = 4p_1p_2\cdots p_r + 3$. By the Fundamental Theorem of Arithmetic, $N$ has some prime divisor. We claim that $N$ has some prime divisor $q_j \in \{4n+3 \mid n\in \Z,\, n \geq 0\}$. Further, we claim that if not, all prime divisors of $N$ is of the form $4n + 1$. This is because we assumed that $q_j$ is a prime divisor and the prime numbers $\geq 3$ not of the form $4n + 3$ can be written as $4n + 1$ for some $n$. And if all prime divisors of $N$ are of the form $4n+1$, then $N$ must also be of the form $4n + 1$ by Lemma \ref{lem:dirichlet-special-case-lem}, which is not true. Then, $q_j$ is either 3 or one of $p_1,\ldots,p_r$.

    If $q_j = 3$, $q_j \divides 3$ and $q_j \divides N$. It follows that $q_j \divides (N-3)$ so $q_j \divides 4 p_1\cdots p_r$. This is a contradiction because $p_1,\ldots,p_r$ are primes not including 3. Hence, $q_j \neq 3$.

    If $q_j \in \{p_1,\ldots,p_r\}$, then $q_j \divides 4p_1\ldots p_r$. And by choice of $q_j$, $q_j \divides N$. It follows that $q_j \divides N - 4p_1\cdots p_r$. This implies $q_j \divides 3$, which is a contradiction as well becuase $3 \not\in \{p_1,\ldots,p_r\}$.

    In both cases, we have a contradiction so the assumption that there are finitely many primes of the form $4n + 3$ must be false.
\end{proof}

Note that this proof will not work for the general case of Dirichlet's theorem because Lemma \ref{lem:dirichlet-special-case-lem} does not hold in the general case.

Let's look another example of a similar special-case proof.

\begin{lemma}
    Let $a,b \in \Z^+$. Suppose $a,b \in \{3n+1 \mid n \in \Z,\, n \geq 0\}$. Then, $ab \in \{3n+1 \mid n \in \Z,\, n \geq 0\}$.
\end{lemma}

\begin{proof}
    Similar to the proof for Lemma \ref{lem:dirichlet-special-case-lem}.
\end{proof}

\begin{theorem}
    There exist infinitely many primes of the form $3n+2$ for $n \geq 0$.
\end{theorem}

\begin{proof}
    Suppose for contradiction that there exist only finitely many primes of the form $3n + 2$. Say there are $r+1$ such primes, namely, $2,p_1,\ldots,p_r$.

    Similar to the proof for the previous theorem, we let
    $$
    N = 3p_1\cdots p_r + 2
    $$
    We claim that there exsits a prime divisor of $N$ of the form $3n+2$. To see why this claim holds, assume that $N$ has no such divisors. Then, there are two possibilities for the prime divisors of $N$. First, we have $3 \divides N$. This is also not possible because $3 \notdivides 2$. This implies that $3n$ is not a prime divisor for $N$. The only remaining possiblity is that all prime divisors of $N$ are of the form $3n + 1$.

    However, by the previous lemma, we know that if all prime divisors of $N$ are of the form $3n + 1$, then $N$ itself must also be of the form $3n + 1$, which is not true. Hence, $N$ must have some prime divisor of the form $3n + 2$. Now, consider the following two cases regarding the prime divisor $q$ of $N$:

    Case 1: $q=2$. we have $2 \divides N$ and clearly $2 \divides 2$. It follows thata $2 \divides N - 2$, but this is a contradiction because $3p_1\cdots p_r$ does not contain 2 as a factor. Therefore, $q=2$ is not possible.

    Case 2: $q \in \{p_1,\ldots,p_r\}$. $q \divides N$ and $q \divides 3p_1\cdots p_r$. It follows that $q \divides 2$. But again, this is not possible becuase $q$ and $2$ are both primes.

    In both cases, we have a contradiction. This implies that $N$ itself is a prime that is not in $\{2,p_1,\ldots,p_r\}$. Hence, our initial assumption that there are finitely many primes of the form $3n + 2$ must be false, so the theorem holds.
\end{proof}

\end{document}