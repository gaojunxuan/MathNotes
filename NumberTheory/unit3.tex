\input{preamble.tex}

\begin{document}
\lecture{MATH453 Elementary Number Theory}{3}{Primes}{Bruce Berndt}{Kevin Gao}

\section{Elementary Properties of Primes}

Recall that a natural number \textbf{greater than 1} is \textit{\textbf{prime}} if it has no factors other than 1 and itself. A natrual number greater than 1 is \textit{\textbf{composite}} if it is not prime.

We begin by introducing some elementary facts about prime numbers.

\begin{lemma} \label{lem:prime-divisor}
    Every integer greater than 1 has a prime divisor.
\end{lemma}

\begin{proof}
    By (strong) induction on $n$.

    \textbf{Base case}: $n = 2$. The lemma clearly holds because $2$ is a prime.

    \textbf{Inductive step}: Let $n \geq 2$ be an arbitrary integer. Suppose that the lemma is true for all integers $2 \leq n' < n$. If $n$ is prime, we are done. So assume $n$ is not prime. Then, by definition, $n$ is composite and can be expressed as $n = ab$ for some $a,b < n$. By induction hypothesis, $a$ and $b$ both have at least one prime divisors. Hence, $n$ also have a prime divisors.
\end{proof}

\begin{theorem}[Infinitude of Primes]
   There exists infinitely many primes. 
\end{theorem}

\begin{proof}
    By contradiction. Suppose there exist only finitely many primes $p_1,\ldots,p_n$.

    Let $N = p_1p_2\ldots p_n + 1$. By Lemma \ref{lem:prime-divisor}, $N$ has at least one prime divisor and since $\{p_1,\ldots,p_n\}$ are all the primes by assumption, there must exists some $i$ such that $p_i \divides N$. Since $p_i \in \{p_1,\ldots,p_n\}$, we have that $p_i \divides p_1\ldots p_n$ trivially. Further, $p_i \divides N$, so $p_i \divides N - p_1\ldots p_n$. This implies that $p_i \divides 1$. But no prime can divide 1. This is a contradiction.
\end{proof}

\begin{proposition}
    If $n$ is composite, then there exists at least one prime $p \leq \sqrt{n}$ dividing $n$.
\end{proposition}

\begin{proof}
    By contradiction. Let $n$ be an arbitrary composite number. By Lemma \ref{lem:prime-divisor}, we know that  has at least one prime divisor $p_j$. Suppose for contradiction that all such $p_j$ are $p_j > \sqrt{n}$.

    $n$ is composite, so we assume that it has $m$ divisors of $n$ where $m \geq 2$. Then,
    $$
    n > \underbrace{\sqrt{n} \sqrt{n} \cdots \sqrt{n}}_{m} = n^{m/2} \geq n
    $$
    This implies $n > n$, which is a contradiction.
\end{proof}

\section{Finding Primes}

Algorithm known as the Sieve\footnote{strainer, colanders, used for filtering; this name is likely due to the fact that the algorithm ``filters out'' the non-primes.} of Eratosthenes.

To find all primes $\leq x$, we list all integers up to $x$. Strike out every integer $\leq \sqrt{x}$ that is a multiple of primes $\leq \sqrt{x}$. In the end, whatever remains are primes.

\begin{example}[Finding primes $\leq 28$]
    List all numbers:

    2 3 4 5 6 7 8 9 10 11 12 13 14 15 16 17 18 19 20 21 22 23 24 25 26 27 28

    2 is prime, so we circle it. Then, we strike out all numbers that is a multiple of 2.

    \boxed{2} 3 \cancel{4} 5 \cancel{6} 7 \cancel{8} 9 \cancel{10} 11 \cancel{12} 13 \cancel{14} 15 \cancel{16} 17 \cancel{18} 19 \cancel{20} 21 \cancel{22} 23 \cancel{24} 25 \cancel{26} 27 \cancel{28}

    The next number, 3, is not struck out, so 3 is prime. We circle 3 and cross out all multiples of 3.

    \boxed{2} \boxed{3} \cancel{4} 5 \cancel{6} 7 \cancel{8} \cancel{9} \cancel{10} 11 \cancel{12} 13 \cancel{14} \cancel{15} \cancel{16} 17 \cancel{18} 19 \cancel{20} \cancel{21} \cancel{22} 23 \cancel{24} 25 \cancel{26} \cancel{27} \cancel{28}

    The next number, 5, is not struck out, so 5 is prime. We circle 5 and cross out all multiples of 5 that are not yet struck out.

    \boxed{2} \boxed{3} \cancel{4} \boxed{5} \cancel{6} 7 \cancel{8} \cancel{9} \cancel{10} 11 \cancel{12} 13 \cancel{14} \cancel{15} \cancel{16} 17 \cancel{18} 19 \cancel{20} \cancel{21} \cancel{22} 23 \cancel{24} \cancel{25} \cancel{26} \cancel{27} \cancel{28}

    Since $4 < \sqrt{28} < 5$, we can stop here and box all the remaining numbers. They are primes because they are not struck out.

    \boxed{2} \boxed{3} \cancel{4} \boxed{5} \cancel{6} \boxed{7} \cancel{8} \cancel{9} \cancel{10} \boxed{11} \cancel{12} \boxed{13} \cancel{14} \cancel{15} \cancel{16} \boxed{17} \cancel{18} 19 \cancel{20} \cancel{21} \cancel{22} \boxed{23} \cancel{24} \cancel{25} \cancel{26} \cancel{27} \cancel{28}
\end{example}

To test if a number $n$ is prime, we test all primes $p \leq \sqrt{n}$. If none divides $n$, then $n$ is prime. There are more efficient algorithms for primality testing. More recently, the AKS primality testing algorithm was shown to be able to run in polynomial time.

\section{Consecutive Composites}

\begin{proposition}
    For every $n \in \Z^+$, there exists $n$ consecutive composite numbers.
\end{proposition}

\begin{proof}
    By construction.

    $(n+1)! + 2$, $(n+1)! + 3$, $\ldots$, $(n+1)! + (n+1)$ are all composite.
\end{proof}

Note that this construction may not give the smallest $n$ consecutive composite numbers.

\end{document}