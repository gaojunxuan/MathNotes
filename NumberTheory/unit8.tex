%
% This is a borrowed LaTeX template file for lecture notes for CS267,
% Applications of Parallel Computing, UCBerkeley EECS Department.
%

\documentclass[twoside]{article}
\usepackage{titlesec}
\setlength{\oddsidemargin}{0.25 in}
\setlength{\evensidemargin}{-0.25 in}
\setlength{\topmargin}{-0.6 in}
\setlength{\textwidth}{6.5 in}
\setlength{\textheight}{8.5 in}
\setlength{\headsep}{0.75 in}
\setlength{\parindent}{0 in}
\setlength{\parskip}{0.1 in}


%
% ADD PACKAGES here:
%

\usepackage{amssymb}	% Already loads amsfonts
\usepackage{amsthm}
\usepackage{graphicx}
\usepackage{mathtools}	% Already loads amsmath
\usepackage{hyperref}
\usepackage{enumitem}
\usepackage{clrscode3e}  % for typesetting pseudocode
\usepackage{ulem}
\usepackage[usenames,dvipsnames]{xcolor}
\usepackage{soul}
\usepackage{cancel}

% Tikz and setup
\usepackage{tikz}
\usepackage{tikz-cd}
\usetikzlibrary{intersections, angles, quotes, calc, positioning}
\usetikzlibrary{arrows.meta}
\usepackage{pgfplots}
\pgfplotsset{compat=1.13}


\tikzset{
    force/.style={thick, {Circle[length=2pt]}-stealth, shorten <=-1pt}
}

%
% The following commands set up the lecnum (lecture number)
% counter and make various numbering schemes work relative
% to the lecture number.
%
\newcounter{lecnum}
\renewcommand{\thepage}{\thelecnum-\arabic{page}}
\renewcommand{\thesection}{\thelecnum.\arabic{section}}
\renewcommand{\theequation}{\thelecnum.\arabic{equation}}
\renewcommand{\thefigure}{\thelecnum.\arabic{figure}}
\renewcommand{\thetable}{\thelecnum.\arabic{table}}

%
% The following macro is used to generate the header.
%
\newcommand{\lecture}[5]{
   \pagestyle{myheadings}
   \thispagestyle{plain}
   \newpage
   \setcounter{lecnum}{#2}
   \setcounter{page}{1}
   \noindent
   \begin{center}
   \framebox{
      \vbox{\vspace{2mm}
    \hbox to 6.28in { {\bf #1
	\hfill} }
       \vspace{4mm}
       \hbox to 6.28in { {\Large \hfill Lecture #2: #3  \hfill} }
       \vspace{2mm}
       \hbox to 6.28in { {\it Lecturer: #4 \hfill Scribe: #5} }
      \vspace{2mm}}
   }
   \end{center}
   \markboth{Lecture #2: #3}{Lecture #2: #3}
   \vspace*{4mm}
}
\renewcommand{\cite}[1]{[#1]}
\def\beginrefs{\begin{list}%
        {[\arabic{equation}]}{\usecounter{equation}
         \setlength{\leftmargin}{2.0truecm}\setlength{\labelsep}{0.4truecm}%
         \setlength{\labelwidth}{1.6truecm}}}
\def\endrefs{\end{list}}
\def\bibentry#1{\item[\hbox{[#1]}]}

\newcommand{\fig}[3]{
			\vspace{#2}
			\begin{center}
			Figure \thelecnum.#1:~#3
			\end{center}
	}

% Colored theorem styles
\makeatother
\usepackage{thmtools}
\usepackage[framemethod=TikZ]{mdframed}
\mdfsetup{skipabove=1em,skipbelow=0.5em}

\declaretheoremstyle[
    headfont=\bfseries\sffamily\color{ForestGreen!70!black}, bodyfont=\normalfont,
    mdframed={
        linewidth=2pt,
        rightline=false, topline=false, bottomline=false,
        linecolor=ForestGreen, backgroundcolor=ForestGreen!5,
    },
    spaceabove=8pt
]{thmgreenbox}

\declaretheoremstyle[
    headfont=\bfseries\sffamily\color{NavyBlue!70!black}, bodyfont=\normalfont,
    mdframed={
        linewidth=2pt,
        rightline=false, topline=false, bottomline=false,
        linecolor=NavyBlue, backgroundcolor=NavyBlue!5,
    },
    spaceabove=8pt
]{thmbluebox}

\declaretheoremstyle[
    headfont=\bfseries\sffamily\color{NavyBlue!70!black}, bodyfont=\normalfont,
    mdframed={
        linewidth=2pt,
        rightline=false, topline=false, bottomline=false,
        linecolor=NavyBlue
    },
    spaceabove=8pt
]{thmblueline}

\declaretheoremstyle[
    headfont=\bfseries\sffamily\color{RawSienna!70!black}, bodyfont=\normalfont,
    mdframed={
        linewidth=2pt,
        rightline=false, topline=false, bottomline=false,
        linecolor=RawSienna, backgroundcolor=RawSienna!5,
    },
    spaceabove=8pt
]{thmredbox}

\declaretheoremstyle[
    headfont=\bfseries\sffamily\color{RawSienna!70!black}, bodyfont=\normalfont,
    numbered=no,
    mdframed={
        linewidth=2pt,
        rightline=false, topline=false, bottomline=false,
        linecolor=RawSienna, backgroundcolor=RawSienna!1,
    },
    qed=\qedsymbol,
    spaceabove=8pt
]{thmproofbox}

\declaretheoremstyle[
    headfont=\bfseries\sffamily\color{NavyBlue!70!black}, bodyfont=\normalfont,
    numbered=no,
    mdframed={
        linewidth=2pt,
        rightline=false, topline=false, bottomline=false,
        linecolor=NavyBlue, backgroundcolor=NavyBlue!1,
    },
    spaceabove=8pt
]{thmexplanationbox}

% Use these for theorems, lemmas, proofs, etc.
\theoremstyle{definition}
\declaretheorem[style=thmgreenbox, name=Definition, numberwithin=lecnum]{definition}
\declaretheorem[style=thmbluebox, numbered=no, name=Example]{example}
\declaretheorem[style=thmredbox, name=Proposition, numberwithin=lecnum]{proposition}
\declaretheorem[style=thmredbox, name=Theorem, numberwithin=lecnum]{theorem}
\declaretheorem[style=thmredbox, name=Lemma, sibling=theorem]{lemma}
\declaretheorem[style=thmredbox, name=Corollary, sibling=theorem]{corollary}
% \newtheorem{theorem}{Theorem}[lecnum]
% \newtheorem{lemma}[theorem]{Lemma}
% \newtheorem{claim}[theorem]{Claim}
% \newtheorem{corollary}[theorem]{Corollary}
% \newtheorem{definition}[theorem]{Definition}
\declaretheorem[style=thmblueline, numbered=no, name=Remark]{remark}
\declaretheorem[style=thmblueline, numbered=no, name=Conjecture]{conjecture}
\renewenvironment{proof}{{\bf \textit{Proof.}}}{\hfill\rule{2mm}{2mm}}
\makeatletter


% **** IF YOU WANT TO DEFINE ADDITIONAL MACROS FOR YOURSELF, PUT THEM HERE:

\renewcommand\Pr{\mathbb{P}}
\newcommand\Ex{\mathbb{E}}

\newcommand\N{\mathbb{N}}
\newcommand\Z{\mathbb{Z}}
\newcommand\Q{\mathbb{Q}}
\newcommand\R{\mathbb{R}}
\newcommand\C{\mathbb{C}}
\newcommand\F{\mathbb{F}}

\DeclarePairedDelimiter\ceil{\lceil}{\rceil}
\DeclarePairedDelimiter\floor{\lfloor}{\rfloor}
\DeclarePairedDelimiter\anglebrac{\langle}{\rangle}

\newcommand{\divides}{\mathrel{\mid}}
\newcommand{\notdivides}{\mathrel{\nmid}}

\newcommand\lcm{\mathrm{lcm}}

\begin{document}
\lecture{MATH453 Elementary Number Theory}{8}{Congruence Classes and Residue System}{Bruce Berndt}{Kevin Gao}

\section{Congruence Classes}

Recall that from last lecture, we defined

\begin{definition}[Congruence Classes]
    The congruence class of $a$ modulo $m$, denoted $[a]_m$, is the set of all integers that are congruent to $a$ modulo $m$
    $$
    \{ z \in \Z \mid m \divides (a-z) \}
    $$
\end{definition}

And we have the proposition

\begin{proposition}
    Let $a,b,c,d \in \Z$. If $a \equiv b \mod m$ and $c \equiv d \mod m$, then
    \begin{equation} \label{eq:congruence-prop-1}
        a + c \equiv b + d \mod m
    \end{equation}
    \begin{equation} \label{eq:congruence-prop-2}
        ac \equiv bd \mod m
    \end{equation}
\end{proposition}

From these two properties, we can define the \textbf{addition} and \textbf{multiplication} operations on congruence classes $(+,\times)$.
$$
[a]_m + [c]_m = [a+c]_m
$$
and
$$
[a]_m \times [c]_m = [ac]_m
$$

Also, recall from last lecture that we cannot just cancel common factors in a congruence relation. We established that in the general case, this will not work. For example, $6 \equiv 3 \mod 3$ but $2 \not\equiv 1 \mod 3$. However, there are cases where we can cancel factors in a congruence.

\begin{proposition} \label{prop:congruence-cancelation}
    $$
    ca \equiv cb \mod m \iff a \equiv b \mod \frac{m}{\gcd(c,m)}
    $$
\end{proposition}

For example, say we have $6 \equiv 3 \mod 3$. By Proposition \ref{prop:congruence-cancelation}, we have $2 \equiv 1 \mod \frac{3}{\gcd(3,3)}$ so $2 \equiv 1 \mod 1$. Now, we prove this proposition.

\begin{proof}

    ($\implies$):
    Assume that $ca \equiv cb \mod m$, which by definition, implies that $m \divides (ca-cb)$ and $m \divides c(a-b)$. By definition of divisibility, there exists some $d$ such that $c(a-b) = md$. Then, we can divide both sides by the greatest common divisor of $c$ and $m$, giving us
    $$
    \frac{c}{\gcd(c,m)} (a-b) = \frac{m}{\gcd(c,m)} d
    $$
    Further, since $\gcd(c,m)$ is the greatest common divisor, $\gcd\left(\frac{c}{\gcd(c,m)},\, \frac{m}{\gcd(c,m)} \right) = 1$. This implies
    $$
    \frac{m}{\gcd(c,m)} \divides (a-b)
    $$
    which by definition means $a \equiv b \mod \frac{m}{\gcd(c,m)}$.

    ($\impliedby$): Assume that $a \equiv b \mod \frac{m}{\gcd(c,m)}$. By definition, $\frac{m}{\gcd(c,m)} \divides (a-b)$. So there exists some $d$ such that
    $$
    a-b = \frac{m}{\gcd(c,m)}d \implies ca - cb = \frac{cm}{\gcd(c,m)} d = \frac{cd}{\gcd(c,m)} m
    $$
    This implies $m \divides (ca-cb)$ since $\frac{cd}{\gcd(c,m)}$ is an integer. Then by definition of congruence, $ca \equiv cb \mod m$.
\end{proof}

\section{Reduced Residue System}

Also recall that from last lecture, we defined a \textit{\textbf{complete residue system}}.

\begin{definition}[Complete Residue System]
    A \textbf{complete residue system} modulo $m$ is a set $S$ of integers such that every $n \in \Z$ is congruent to one and only one member of $S$.
\end{definition}

\begin{definition}[Reduced Residue System]
    A \textbf{reduced residue system} modulo $m$ is a set of integers $r_1, \ldots, r_n$ such that if $\gcd(a,m) = 1$, then $a \equiv r_j \mod m$ for one and only one value of $j$.
\end{definition}

Stated slightly differently, a reduced residue system modulo $m$ is a set of integers $r_i$ such that $\gcd(r_i,m) = 1$ for all $i$, and $r_i \not\equiv r_j \mod m$ for all $j \neq i$. That is, each element in a reduced residue system is relatively prime to $m$ and no two elements of the set are congruent modulo $m$.

Note that the definition of a reduced residue system immediately implies that $n < m$. To see why, suppose $n = m$ and we have a complete residue system. Then, $m \equiv m \mod m$. WLOG, suppose $m = r_j$ for some $j$ (otherwise, we can choose $r_j$ to be some multiple of $m$). By definition, there's some $a$ such that $a \equiv m \mod m$ but this is impossible since $a$ and $m$ are relatively prime by definition of a reduced residue system. This implies that $m$ or any multiple of $m$ must not be an element in a reduced residue system.

Another way of looking at a reduced residue system is that we can take a complete residue system, remove certain numbers, and get back a reduced residue system. In particular, if we have a complete residue system modulo $m$, and we remove all $r_j$ such that $\gcd(r_j, m) = 1$, the resulting system is a reduced residue system. This should be clear from the definition of a reduced system.

Additionally, if $\gcd(a,m) = 1$ and $a \equiv r_j \mod m$ for some $a$, then $\gcd(r_j,m) = 1$. This essentially shows that our alternative definition is the same as the original definition.

\begin{proof}
    Suppose not. That is, there exists $a$ such that $\gcd(a,m) = 1$ and $m \divides (a-r_j)$ but $\gcd(r_j,m) \neq 1$. This implies there exists some $p$ such that $p \divides r_j$ and $p \divides m$. But we also have $a - r_j = md$ for some $d$ since $m \divides (a-r_j)$. This implies $p \divides a$. But by our assumption, $a$ and $m$ should be relatively prime, so this is a contradiction.
\end{proof}

\section{Euler's Phi Function}

The number of elements in a reduced residue system modulo $m$ for some fixed $m$ is \textbf{constant}. We call this number \textit{\textbf{Euler's phi function}} or \textit{\textbf{Euler's totient function}}. The Euler's phi function for $m$ is denoted by
$$
\varphi(m)
$$

\begin{theorem}
    Let $r_1,\ldots,r_n$ be a complete/reduced residue system modulo $m$. Let $\gcd(a,m)=1$. Then,
    $$
    \{ ar_1,\ldots, ar_n \}
    $$
    is still a complete/reduced residue system modulo $m$.
\end{theorem}

\begin{proof}
    Suppose for contradiction that $\{ar_1,\ldots,ar_n\}$ is not a complete/reduced residue system modulo $m$ for some $m$. Then, there must exitsts some $i$ and $j$ such that $ar_i \equiv ar_j \mod m$ (if no such $i,j$ exists, then $\{ar_1,\ldots,ar_n\}$ would indeed be complete/reduced). But since $\gcd(a,m) = 1$, $ar_i \equiv ar_j \mod m \iff r_i \equiv r_j \mod m$. This is a contradiction to the assumption that $\{r_1,\ldots,r_n\}$ is a complete/reduced residue system.
\end{proof}

\end{document}