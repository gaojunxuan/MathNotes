\section{Linear Congruence}

We are all familiar with linear equations like
$$
ax + b = 0
$$
We are going to talk about a different kind of linear ``equation'' known as linear congruence. They are of the form
$$
ax \equiv b \mod m
$$
Let's consider some linear congruence. First, we look at $3x \equiv 1 \mod 6$. If we try everything from 0 to 5, it is easy to notice that this does not have a solution.

How about $2x \equiv 4 \mod 6$. We have one solution $\{2,8,14,\ldots,\}$ and another solution $\{5,11,17,\ldots\}$. In total, we have two solutions. Further, these two solution sets are incongruent because elements from one set is not congruent to those from the other set modulo $6$.

$2x \equiv 5 \mod 6$ does not have a solution. But $3x \equiv 1 \mod 5$ have the solution $\{2,7,12,\ldots,\}$.

The two examples that do not have a solution are $3x \equiv 1 \mod 6$ and $2x \equiv 5 \mod 6$. The ones that do have solutions are $2x \equiv 4 \mod 6$ and $3x \equiv 1 \mod 5$. From these examples, we make the observation that one thing in common among the linear congruences that do not have a solution is that the GCD of $a$ and $m$ does not divide $b$. In particular, we have that $\gcd(3,6) \notdivides 1$ and $\gcd(2,6) \notdivides 5$. On the other hand, $\gcd(2,6) \divides 4$ and $\gcd(3,5) \divides 1$.

We can generalize this into the following theorem:

\begin{theorem} \label{thm:lin-congruence-soln}
    Let $ax \equiv b \mod m$ be a linear congruence in one variable and let $d = \gcd(a,m)$. If $d \notdivides b$, then the linear congruence has no solution in $\Z$. If $d \divides b$, then the linear congruence has exactly $d$ incongruent solutions modulo $m$ in $\Z$.
\end{theorem}

The proof can be broken into two parts: 

(1) Showing that if $d \notdivides b$, then the linear congruence has no solution. This can be proved using the contrapositive of the original statement (namely, if the linear congruence has solution, then $d \divides b$); 

(2) Showing that if $d \divides b$, then the linear congruence has a solution, $x_0$. And given a solution $x_0$, we can construct infinitely many solutions of a given form and that among those infinitely many solutions, we have $d$ incongruent solutions. More specifically, it suffices to show the following:

\begin{addmargin}[1em]{0em}
    a. Show that a solution $x_0$ exists. \\
    b. Given a solution $x_0$, show that $ax \equiv b \mod m$ has infinitely many solutions in $\Z$ of a given form. \\
    c. Given a solution $x_0$, show that every solution has the form in (b). \\
    d. Show that there are $d$ incongruent solutions.
\end{addmargin}

\begin{proof}
    We begin the proof by proving \textbf{Part (1)} by its \textbf{contrapositive}.
    
    Assume that $ax \equiv b \mod m$ has a solution. By definition of congruence, $m \divides ax - b$. By definition of divisibility, $m \divides ax - b$ iff there exists some $y \in \Z$ such that $my = ax-b$. This, in turn, is true iff $ax - my = b$ has a solution. Since $d$ is the gcd of $a$ and $m$, $d \divides a$ and $d \divides m$. By Proposition 2.2, it follows that $d \divides ax - my$. Since $b = ax - my$ iff $ax \equiv b \mod m$ has a solution, $d \divides b$ if the original linear congruence has a solution.

    Now we proceed to prove \textbf{Part (2)a}. Assume $d \divides b$. Since $d$ is the \textbf{gcd} of $a$ and $m$, by \textbf{Proposition \ref{prop:gcd-linear-combination}}, there exists $r,s \in \Z$ such that
    $$
    d = ar + ms
    $$
    Further, $d \divides b$ implies $b = de$ for some $e \in \Z$. So, by substitution
    $$
    b = de = (ar + ms)e = a(re) + m(se)
    $$
    which clearly suggests a solution with $x = re$ and $y = -se$ that solves $ax - my = b$ (and thus solves $ax \equiv b \mod m$).

    For \textbf{Part (2)b}, let $x_0$ be an arbitrary solution for the linear congruence $ax \equiv b \mod m$. Let $n \in \Z$ and we consider
    $$
    x = x_0 + \left( \frac{m}{d} \right) n
    $$
    Since $d \divides m$, $\frac{m}{d}$ is an integer, so it follows that $x$ is also an integer. Furthermore, we observe that
    $$
    \begin{aligned}
        a\left( x_0 + \left( \frac{m}{d} \right) n \right) &= ax_0 + a\left( \frac{m}{d} \right) n \\
        &= ax_0 + \left( \frac{a}{d} \right) mn \\
        &\equiv ax_0 \mod m \\
        &\equiv b \mod m
    \end{aligned}
    $$
    Since for every solution $x$, $ax \equiv b \mod m$ but also $b \equiv a(x_0 + (\frac{m}{d})n) \mod m$, for all $n \in \Z$,
    $$
    x_0 + \left( \frac{m}{d} \right) n
    $$
    is also a solution to $ax \equiv b \mod m$. It follows that given any solution $x_0$ of $ax \equiv b \mod m$, there are infinitely many solutions of the form $x_0 + (m/d)n$ for $n \in \Z$.

    For \textbf{Part 2(c)}, let $x_0$ be a solution of $ax \equiv b \mod m$. Then, $ax_0 - my_0 = b$ for some $y_0 \in \Z$. Now, any other solution $x$ of $ax \equiv b \mod m$ implies the existence of $y \in \Z$ with $ax - my = b$, and we have
    $$
    (ax - my) - (ax_0 - my_0) = b - b = 0
    $$
    and
    $$
    a(x-x_0) = m(y-y_0)
    $$
    Dividing both sides by $d$, we have
    $$
    \left( \frac{a}{d} \right) (x-x_0) = \left( \frac{m}{d} \right) (y-y_0)
    $$
    Now $\frac{m}{d}\divides (\frac{a}{d}) (x-x_0)$. Since $\gcd(\frac{a}{d}, \frac{m}{d}) = 1$, we have $\frac{m}{d} \divides x-x_0$. Consequently, $x-x_0 = (\frac{m}{d}) n$ for some $n \in \Z$. Equivalently, $x = x_0 + (\frac{m}{d})n$ for some $n \in \Z$. Combining this result with 2(b), we hae shown that all solutions of $ax \equiv b \mod m$ are given precisely by $x_0 + (\frac{m}{d})n$ for $n \in \Z$.
    
    For \textbf{2(d)}, to determine how many incongruent solutions modulo $m$ exist among the solutions of the form $x_0 + (\frac{m}{d})n$ for $n \in \Z$, we establish a \textbf{necessary and sufficient condition} (chain of iff.) for the congruence modulo $m$ of two such solutions. Consider
    $$
    x_0 + \left( \frac{m}{d} \right) n_1 \equiv x_0 + \left( \frac{m}{d} \right) n_2 \mod m
    $$
    if and only if
    $$
    \left( \frac{m}{d} \right) n_1 \equiv \left( \frac{m}{d} \right) n_2 \mod m 
    $$
    if and only if
    $$
    m \divides \left( \frac{m}{d} \right) (n_1-n_2)
    $$
    if and only if
    $$
    d \divides n_1 - n_2
    $$
    if and only if
    $$
    n_1 \equiv n_2 \mod d
    $$
    Therefore, two solutions of the form $x_0 + (\frac{m}{d}) n$ are congruent modulo $m$ if and only if the $n$-values of these two solutions are congruent modulo $d$. Thus, a complete set of incongruent solutions modulo $m$ of $ax \equiv b \mod m$ can be obtained from an initial solution $x_0 + (\frac{m}{d})n$ by letting $n$ range over a complete residue system modulo $d$, that is $\{0,1,2,\ldots,d-1\}$.
\end{proof}

\begin{corollary} \label{cor:all-congruence-solutions}
    Let $ax \equiv b \mod m$ be a linear congruence in one variable and let $d = \gcd(a,m)$. If $d \divides b$, then there are $d$ incongruent solutions modulo $m$ given by
    $$
    x_0 + \left( \frac{m}{d} \right) n \qquad n = 0,1\ldots,d-1
    $$
    where $x_0$ is a particular solution of the congruence.
\end{corollary}

\section{Solving Linear Congruence}

It is perfectly acceptable to solve linear congruence by inspection (guess and check). However, for larger $m$, it might be too slow and tedious to solve the linear congruence simply by inspection. We now introduce a general procedure for solving linear congruence.

Recall that a linear congruence $ax \equiv b \mod m$ can always be expressed as $ax - my = b$ for some $y \in \Z$. We begin by considering an example:
$$
9x \equiv 21 \mod 30
$$
which is equivalent to $21 = 9x - 30y$. Since $\gcd(9,30) \divides 21$ so the linear congruence has solutions. We apply the Euclidean algorithm
$$
\begin{aligned}
    30 &= 3 \cdot 9 + \boxed{3} \\
    9 &= 3 \cdot 3 + 0
\end{aligned}
$$
So we know that $\gcd(9,30) = 3 = 1 \cdot 30 - 3 \cdot 9$ (recall that another way to look at gcd is as the minimum positive linear combination of the two numbers). Then, multiply both sides of the equation by 7 so that the LHS is equal to 21, and we have
$$
21 = (1 \cdot 7) 30 - (3 \cdot 7) \cdot 9
$$
Comparing this with the equation equivalent to our congruence equation, $21 = 9x - 30y$, we know that $x = -21$ and $y = -7$ is a solution. To get other olutions, we apply Corollary \ref{cor:all-congruence-solutions}, giving us $x = -21 + 10 = -11$, $x = -21 + 20 = -1$, $x = -21 + 30 = 9$, etc. Note that once we go above -1, all the other solutions may no longer be incongruent to the previous solutions.

\begin{example}[A harder example]
    Solve
    $$
    481x \equiv 627 \mod 703
    $$
\end{example}
Use the Euclidean algorithm to find $d = \gcd(481, 703)$ 
$$
\begin{aligned}
    703 &= 1 \cdot 481 + 222 \\
    481 &= 2 \cdot 222 + \boxed{37} \\
    222 &= 6 \cdot 37 + 0
\end{aligned}
$$
Since $d = 37$ does not divide 627, we are done and the linear congruence does not have a solution.

\begin{example}[A harder example (that actually has a solution)]
    Solve
    $$
    481x \equiv 629 \mod 703
    $$
\end{example}
Use the Euclidean algorithm to find $d = \gcd(481, 703)$ 
$$
\begin{aligned}
    703 &= 1 \cdot 481 + 222 \\
    481 &= 2 \cdot 222 + \boxed{37} \\
    222 &= 6 \cdot 37 + 0
\end{aligned}
$$
This time, $37 \divides 629$ so there are 37 incongruent solutions modulo $307$. Next, we express $d = 37$ as a linear combination of 481 and 703.
$$
\begin{aligned}
    37 &= 1 \cdot 481 - 2 \cdot 222 \\
    &= 1 \cdot 481 - 2 \cdot (703 - 1 \cdot 481) \\
    &= 3 \cdot 481 - 2 \cdot 703
\end{aligned}
$$
Multiply both sides by 17 and
$$
629 = (3 \cdot 17) \cdot 481 - (2 \cdot 17) \cdot 703
$$
so $x = 51$ is a solution. The remaining incongruent solutions can be obtained with
$$
51 + \frac{703}{37}n \qquad n = 0,1,\ldots,36
$$