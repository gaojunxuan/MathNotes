%
% This is a borrowed LaTeX template file for lecture notes for CS267,
% Applications of Parallel Computing, UCBerkeley EECS Department.
%

\documentclass{article}
\usepackage{titlesec}
%\setlength{\oddsidemargin}{0.25 in}
%\setlength{\evensidemargin}{-0.25 in}
\setlength{\oddsidemargin}{0 in}
\setlength{\evensidemargin}{0 in}
\setlength{\topmargin}{-0.6 in}
\setlength{\textwidth}{6.5 in}
\setlength{\textheight}{8.5 in}
\setlength{\headsep}{0.75 in}
\setlength{\parindent}{0 in}
\setlength{\parskip}{0.1 in}


%
% ADD PACKAGES here:
%

\usepackage{amssymb}	% Already loads amsfonts
\usepackage{amsthm}
\usepackage{graphicx}
\usepackage{mathtools}	% Already loads amsmath
\usepackage{hyperref}
\usepackage{enumitem}
\usepackage{clrscode3e}  % for typesetting pseudocode
\usepackage{ulem}
\usepackage[usenames,dvipsnames]{xcolor}
\usepackage{multicol}


% Tikz and setup
\usepackage{tikz}
\usepackage{tikz-cd}
\usetikzlibrary{intersections, angles, quotes, calc, positioning}
\usetikzlibrary{arrows.meta}
\usepackage{pgfplots}
\pgfplotsset{compat=1.13}


\tikzset{
    force/.style={thick, {Circle[length=2pt]}-stealth, shorten <=-1pt}
}

%
% The following commands set up the lecnum (lecture number)
% counter and make various numbering schemes work relative
% to the lecture number.
%
\newcounter{lecnum}
\renewcommand{\thepage}{\thelecnum-\arabic{page}}
\renewcommand{\thesection}{\thelecnum.\arabic{section}}
\renewcommand{\theequation}{\thelecnum.\arabic{equation}}
\renewcommand{\thefigure}{\thelecnum.\arabic{figure}}
\renewcommand{\thetable}{\thelecnum.\arabic{table}}

%
% The following macro is used to generate the header.
%
\newcommand{\lecture}[5]{
   \pagestyle{myheadings}
   \thispagestyle{plain}
   \newpage
   \setcounter{lecnum}{#2}
   \setcounter{page}{1}
   \noindent
   \begin{center}
   \framebox{
      \vbox{\vspace{2mm}
    \hbox to 6.28in { {\bf #1
	\hfill} }
       \vspace{4mm}
       \hbox to 6.28in { {\Large \hfill Lecture #2: #3  \hfill} }
       \vspace{2mm}
       \hbox to 6.28in { {\it Lecturer: #4 \hfill Scribe: #5} }
      \vspace{2mm}}
   }
   \end{center}
   \markboth{Lecture #2: #3}{Lecture #2: #3}
   \vspace*{4mm}
}
\renewcommand{\cite}[1]{[#1]}
\def\beginrefs{\begin{list}%
        {[\arabic{equation}]}{\usecounter{equation}
         \setlength{\leftmargin}{2.0truecm}\setlength{\labelsep}{0.4truecm}%
         \setlength{\labelwidth}{1.6truecm}}}
\def\endrefs{\end{list}}
\def\bibentry#1{\item[\hbox{[#1]}]}

\newcommand{\fig}[3]{
			\vspace{#2}
			\begin{center}
			Figure \thelecnum.#1:~#3
			\end{center}
	}

% Colored theorem styles
\makeatother
\usepackage{thmtools}
\usepackage[framemethod=TikZ]{mdframed}
\mdfsetup{skipabove=1em,skipbelow=1em}

\declaretheoremstyle[
    headfont=\bfseries\sffamily\color{ForestGreen!70!black}, bodyfont=\normalfont,
    mdframed={
        linewidth=2pt,
        rightline=false, topline=false, bottomline=false,
        linecolor=ForestGreen, backgroundcolor=ForestGreen!5,
    },
    spaceabove=8pt
]{thmgreenbox}

\declaretheoremstyle[
    headfont=\bfseries\sffamily\color{NavyBlue!70!black}, bodyfont=\normalfont,
    mdframed={
        linewidth=2pt,
        rightline=false, topline=false, bottomline=false,
        linecolor=NavyBlue, backgroundcolor=NavyBlue!5,
    },
    spaceabove=8pt
]{thmbluebox}

\declaretheoremstyle[
    headfont=\bfseries\sffamily\color{NavyBlue!70!black}, bodyfont=\normalfont,
    mdframed={
        linewidth=2pt,
        rightline=false, topline=false, bottomline=false,
        linecolor=NavyBlue
    },
    spaceabove=8pt
]{thmblueline}

\declaretheoremstyle[
    headfont=\bfseries\sffamily\color{RawSienna!70!black}, bodyfont=\normalfont,
    mdframed={
        linewidth=2pt,
        rightline=false, topline=false, bottomline=false,
        linecolor=RawSienna, backgroundcolor=RawSienna!5,
    },
    spaceabove=8pt
]{thmredbox}

\declaretheoremstyle[
    headfont=\bfseries\sffamily\color{RawSienna!70!black}, bodyfont=\normalfont,
    numbered=no,
    mdframed={
        linewidth=2pt,
        rightline=false, topline=false, bottomline=false,
        linecolor=RawSienna, backgroundcolor=RawSienna!1,
    },
    qed=\qedsymbol,
    spaceabove=8pt
]{thmproofbox}

\declaretheoremstyle[
    headfont=\bfseries\sffamily\color{NavyBlue!70!black}, bodyfont=\normalfont,
    numbered=no,
    mdframed={
        linewidth=2pt,
        rightline=false, topline=false, bottomline=false,
        linecolor=NavyBlue, backgroundcolor=NavyBlue!1,
    },
    spaceabove=8pt
]{thmexplanationbox}

% Use these for theorems, lemmas, proofs, etc.
\theoremstyle{definition}
\declaretheorem[style=thmgreenbox, name=Definition, numberwithin=lecnum]{definition}
\declaretheorem[style=thmbluebox, numbered=no, name=Example]{example}
\declaretheorem[style=thmredbox, name=Theorem, numberwithin=lecnum]{theorem}
\declaretheorem[style=thmredbox, name=Proposition, sibling=theorem]{proposition}
\declaretheorem[style=thmredbox, name=Lemma, sibling=theorem]{lemma}
\declaretheorem[style=thmredbox, name=Corollary, sibling=theorem]{corollary}
% \newtheorem{theorem}{Theorem}[lecnum]
% \newtheorem{lemma}[theorem]{Lemma}
% \newtheorem{claim}[theorem]{Claim}
% \newtheorem{corollary}[theorem]{Corollary}
% \newtheorem{definition}[theorem]{Definition}
\declaretheorem[style=thmblueline, numbered=no, name=Remark]{remark}
\renewenvironment{proof}{{\bf \textit{Proof.}}}{\hfill\rule{2mm}{2mm}}
\makeatletter


% **** IF YOU WANT TO DEFINE ADDITIONAL MACROS FOR YOURSELF, PUT THEM HERE:

\renewcommand\Pr{\mathbb{P}}
\newcommand\Ex{\mathbb{E}}

\newcommand\N{\mathbb{N}}
\newcommand\Z{\mathbb{Z}}
\newcommand\Q{\mathbb{Q}}
\newcommand\R{\mathbb{R}}
\newcommand\C{\mathbb{C}}
\newcommand\F{\mathbb{F}}

\DeclarePairedDelimiter\ceil{\lceil}{\rceil}
\DeclarePairedDelimiter\floor{\lfloor}{\rfloor}
\DeclarePairedDelimiter\anglebrac{\langle}{\rangle}

\begin{document}
\lecture{MAT344 Intro to Combinatorics}{3}{Binomial Theorem}{Reila Zheng}{Kevin Gao}

\section{Binomial Theorem}

\begin{theorem}[Binomial Theorem]
    For any $x \in \R$, for any $n \geq 0$ natural number
    $$
    (1+x)^n = \sum_{k=0}^n \binom{n}{k} x^k
    $$
\end{theorem}

\begin{proof}
    $$
    \text{LHS} = (1+x)^n = \underbrace{(1+x) (1+x) \cdots (1+x)}_{\text{$n$ times}}
    $$
    When we expand and collect the like terms, we get a polynomial of the form
    $$
    \sum_{k=0}^n c_k x^k = c_0 + c_1x + c_2 x^2 + \cdots + c_n x^n
    $$
    For $k$, the only way to get $x^k$ in the expansion of the LHS is if for $k$ of the $n$ terms in the product to contribute to to $x$, and the rest $n-k$ of the terms to contribute to $1$.

    In total, we have $\binom{n}{k}$ ways to get $x^k$ in the expansions. Hence, $c_k = \binom{n}{k}$.
\end{proof}

The binomial theorem can be equivalently stated as
\begin{theorem}[Binomial Theorem (two variables)]
    For any $x,y \in \R$, $n \in \N$ such that $n \geq 0$,
    $$
    (x+y)^n = \sum_{k=0}^k \binom{n}{k} x^k y^{n-k}
    $$
\end{theorem}

\begin{example}
    $$
    (1+x)^2 = (1 + x)(1 + x) = 1 + \underbrace{x + x}_{\substack{\text{two ways of} \\  \text{picking one 1} \\ \text{and one x}}} + x^2
    $$
    $$
    (1+x)^3 = (1+x) (1+x) (1+x) = 1 + \underbrace{x + x + x}_{\binom{3}{1}x} + \binom{3}{2} x^2 + \binom{3}{3}x^3
    $$
    Choose 3 to be $x$ and other 0 to be 1.
\end{example}

\section{Combinatorial Proofs}

Say you want to prove that
$$
2^n = \sum_{k=0}^n \binom{n}{k}
$$
Combinatorially, the LHS is the number of binary strings of length $n$. The RHS is obtained by summing up the number of binary strings with $k$ 0's in the binary string of length $n$ over all $k = \{0,\ldots,n\}$.

\begin{remark}
    The hard part is making sure you count all cases and are not double counting.
\end{remark}

\begin{example}
    Say we want to prove the identity
    $$
    \binom{2n}{n} = \sum_{k=0}^n \binom{n}{k} \binom{n}{n-k}
    $$
    The LHS can be described as follows. Suppose we have a group of $n$ boys and $n$ girls. In total, we have $2n$ people. Choose $n$ people to form a team.

    For the RHS, we can split into cases where $k$ girls are chosen. For each $k$, there are $\binom{n}{k}$ ways to choose girls and $\binom{n}{n-k}$ ways to choose $n-k$ boys. This summed over all $k \in \{0,\ldots,n\}$ gives us $\sum_{k=0}^n \binom{n}{k} \binom{n}{n-k}$.
\end{example}

\begin{remark}
    Combinatorially, summation often means summing up the counts for each case.
\end{remark}

\subsection{Ideas of a Combinatorial Proof}

Say you want to show an identity with LHS = RHS using a combinatorial proof. We follow these steps
\begin{enumerate}
    \item Come up with a situation to count one side (whichever is easier).
    \item Come up with another way to count the same situation, as described by the harder side.
    \item Show that both count the same objects and conclude that LHS = RHS.
\end{enumerate}

\begin{remark}
    Some hints for writing combinatorial proofs:
    \begin{itemize}
        \item Start with the easier side
        \item Break down the hard side (especially those with summations) into cases based on $k$ and note that sum ($\Sigma$) = ``OR'', product ($\Pi$) = ``AND''
        \item Use facts like $\binom{n}{k} = \binom{n}{n-k}$, $\binom{n}{1} = n$, etc.
        \item There might be information not captured in the algebraic identity
    \end{itemize}
\end{remark}

\begin{example}
    $$
    \sum_{k=1}^n \binom{n}{k} k = n 2^{n-1}
    $$
    for $n \geq 1$.

    For RHS, there are $n$ people and we want to choose 1 to be the captain ($\binom{n}{1}$). Build a team around the captain. For each of the $n-1$ others not chosen, they can be either on the team or not. Thus, $n (2^{n-1})$.

    For LHS, we split into cases based on there being $k$ people on the team. First, choose $k$ of the $n$ people to be on the team. From the $k$ people, we choose 1 to be the captain ($\binom{k}{1}$). Summing this for all $k \in \{1,\ldots,n\}$ gives the number of all possible configuraitons to form a team with at least one captain.
\end{example}

\end{document}