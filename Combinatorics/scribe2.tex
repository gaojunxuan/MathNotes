%
% This is a borrowed LaTeX template file for lecture notes for CS267,
% Applications of Parallel Computing, UCBerkeley EECS Department.
%

\documentclass{article}
\usepackage{titlesec}
%\setlength{\oddsidemargin}{0.25 in}
%\setlength{\evensidemargin}{-0.25 in}
\setlength{\oddsidemargin}{0 in}
\setlength{\evensidemargin}{0 in}
\setlength{\topmargin}{-0.6 in}
\setlength{\textwidth}{6.5 in}
\setlength{\textheight}{8.5 in}
\setlength{\headsep}{0.75 in}
\setlength{\parindent}{0 in}
\setlength{\parskip}{0.1 in}


%
% ADD PACKAGES here:
%

\usepackage{amssymb}	% Already loads amsfonts
\usepackage{amsthm}
\usepackage{graphicx}
\usepackage{mathtools}	% Already loads amsmath
\usepackage{hyperref}
\usepackage{enumitem}
\usepackage{clrscode3e}  % for typesetting pseudocode
\usepackage{ulem}
\usepackage[usenames,dvipsnames]{xcolor}
\usepackage{multicol}


% Tikz and setup
\usepackage{tikz}
\usepackage{tikz-cd}
\usetikzlibrary{intersections, angles, quotes, calc, positioning}
\usetikzlibrary{arrows.meta}
\usepackage{pgfplots}
\pgfplotsset{compat=1.13}


\tikzset{
    force/.style={thick, {Circle[length=2pt]}-stealth, shorten <=-1pt}
}

%
% The following commands set up the lecnum (lecture number)
% counter and make various numbering schemes work relative
% to the lecture number.
%
\newcounter{lecnum}
\renewcommand{\thepage}{\thelecnum-\arabic{page}}
\renewcommand{\thesection}{\thelecnum.\arabic{section}}
\renewcommand{\theequation}{\thelecnum.\arabic{equation}}
\renewcommand{\thefigure}{\thelecnum.\arabic{figure}}
\renewcommand{\thetable}{\thelecnum.\arabic{table}}

%
% The following macro is used to generate the header.
%
\newcommand{\lecture}[5]{
   \pagestyle{myheadings}
   \thispagestyle{plain}
   \newpage
   \setcounter{lecnum}{#2}
   \setcounter{page}{1}
   \noindent
   \begin{center}
   \framebox{
      \vbox{\vspace{2mm}
    \hbox to 6.28in { {\bf #1
	\hfill} }
       \vspace{4mm}
       \hbox to 6.28in { {\Large \hfill Lecture #2: #3  \hfill} }
       \vspace{2mm}
       \hbox to 6.28in { {\it Lecturer: #4 \hfill Scribe: #5} }
      \vspace{2mm}}
   }
   \end{center}
   \markboth{Lecture #2: #3}{Lecture #2: #3}
   \vspace*{4mm}
}
\renewcommand{\cite}[1]{[#1]}
\def\beginrefs{\begin{list}%
        {[\arabic{equation}]}{\usecounter{equation}
         \setlength{\leftmargin}{2.0truecm}\setlength{\labelsep}{0.4truecm}%
         \setlength{\labelwidth}{1.6truecm}}}
\def\endrefs{\end{list}}
\def\bibentry#1{\item[\hbox{[#1]}]}

\newcommand{\fig}[3]{
			\vspace{#2}
			\begin{center}
			Figure \thelecnum.#1:~#3
			\end{center}
	}

% Colored theorem styles
\makeatother
\usepackage{thmtools}
\usepackage[framemethod=TikZ]{mdframed}
\mdfsetup{skipabove=1em,skipbelow=1em}

\declaretheoremstyle[
    headfont=\bfseries\sffamily\color{ForestGreen!70!black}, bodyfont=\normalfont,
    mdframed={
        linewidth=2pt,
        rightline=false, topline=false, bottomline=false,
        linecolor=ForestGreen, backgroundcolor=ForestGreen!5,
    },
    spaceabove=8pt
]{thmgreenbox}

\declaretheoremstyle[
    headfont=\bfseries\sffamily\color{NavyBlue!70!black}, bodyfont=\normalfont,
    mdframed={
        linewidth=2pt,
        rightline=false, topline=false, bottomline=false,
        linecolor=NavyBlue, backgroundcolor=NavyBlue!5,
    },
    spaceabove=8pt
]{thmbluebox}

\declaretheoremstyle[
    headfont=\bfseries\sffamily\color{NavyBlue!70!black}, bodyfont=\normalfont,
    mdframed={
        linewidth=2pt,
        rightline=false, topline=false, bottomline=false,
        linecolor=NavyBlue
    },
    spaceabove=8pt
]{thmblueline}

\declaretheoremstyle[
    headfont=\bfseries\sffamily\color{RawSienna!70!black}, bodyfont=\normalfont,
    mdframed={
        linewidth=2pt,
        rightline=false, topline=false, bottomline=false,
        linecolor=RawSienna, backgroundcolor=RawSienna!5,
    },
    spaceabove=8pt
]{thmredbox}

\declaretheoremstyle[
    headfont=\bfseries\sffamily\color{RawSienna!70!black}, bodyfont=\normalfont,
    numbered=no,
    mdframed={
        linewidth=2pt,
        rightline=false, topline=false, bottomline=false,
        linecolor=RawSienna, backgroundcolor=RawSienna!1,
    },
    qed=\qedsymbol,
    spaceabove=8pt
]{thmproofbox}

\declaretheoremstyle[
    headfont=\bfseries\sffamily\color{NavyBlue!70!black}, bodyfont=\normalfont,
    numbered=no,
    mdframed={
        linewidth=2pt,
        rightline=false, topline=false, bottomline=false,
        linecolor=NavyBlue, backgroundcolor=NavyBlue!1,
    },
    spaceabove=8pt
]{thmexplanationbox}

% Use these for theorems, lemmas, proofs, etc.
\theoremstyle{definition}
\declaretheorem[style=thmgreenbox, name=Definition, numberwithin=lecnum]{definition}
\declaretheorem[style=thmbluebox, numbered=no, name=Example]{example}
\declaretheorem[style=thmredbox, name=Theorem, numberwithin=lecnum]{theorem}
\declaretheorem[style=thmredbox, name=Proposition, sibling=theorem]{proposition}
\declaretheorem[style=thmredbox, name=Lemma, sibling=theorem]{lemma}
\declaretheorem[style=thmredbox, name=Corollary, sibling=theorem]{corollary}
% \newtheorem{theorem}{Theorem}[lecnum]
% \newtheorem{lemma}[theorem]{Lemma}
% \newtheorem{claim}[theorem]{Claim}
% \newtheorem{corollary}[theorem]{Corollary}
% \newtheorem{definition}[theorem]{Definition}
\declaretheorem[style=thmblueline, numbered=no, name=Remark]{remark}
\renewenvironment{proof}{{\bf \textit{Proof.}}}{\hfill\rule{2mm}{2mm}}
\makeatletter


% **** IF YOU WANT TO DEFINE ADDITIONAL MACROS FOR YOURSELF, PUT THEM HERE:

\renewcommand\Pr{\mathbb{P}}
\newcommand\Ex{\mathbb{E}}

\newcommand\N{\mathbb{N}}
\newcommand\Z{\mathbb{Z}}
\newcommand\Q{\mathbb{Q}}
\newcommand\R{\mathbb{R}}
\newcommand\C{\mathbb{C}}
\newcommand\F{\mathbb{F}}

\DeclarePairedDelimiter\ceil{\lceil}{\rceil}
\DeclarePairedDelimiter\floor{\lfloor}{\rfloor}
\DeclarePairedDelimiter\anglebrac{\langle}{\rangle}