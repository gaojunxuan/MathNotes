\section{Ordinary Generating Function}

\begin{definition}[Ordinary Generating Function]
    Given a sequence of real numbers $a_n$ for $n \geq 0$, we call the function
    $$
    F(x) = \sum_{n=0}^\infty a_n x^n
    $$
    the \textbf{ordinary generating function} associated to $a_n$.
\end{definition}

The above may not be convergent for any $x$ although often in combinatorics, it will be convergent for many $x$. We mainly use these functions as a bookkeeping system. Generating function is a useful tool to analyze the behavior of sequences.

\subsection{Geometric Series}

\begin{theorem}
    Consider the sequence $a_n = 1$ for all integers $n \geq 0$. The associated generating function is $F(x) = \frac{1}{1-x}$.
\end{theorem}

\begin{proof}
    It suffices to show that $(1-x) F(x) = 1$. Consider the expansion of this expression:
    $$
    \begin{aligned}
        (1-x) F(x) &= (1-x) \sum_{n=0}^\infty x^n \\
        &= \sum_{n=0}^\infty x^n - x \sum_{n=0}^\infty x^n \\
        &= \sum_{n=0}^\infty x^n - \sum_{n=0}^\infty x^{n+1} \\
        &= 1
    \end{aligned}
    $$
\end{proof}

Next, we consider the \textbf{finite geometric series}.

\begin{theorem}
    Consider the sequence $a_n = 1$ for all integers $0 \leq n \leq k$ and $a_n = 0$ when $k > n$. The associated generating function is $F(x) = \frac{1-x^{k+1}}{1-x}$
\end{theorem}

\begin{proof}
    $$
    \begin{aligned}
        (1-x)F(x) &= (1-x) \sum_{n=0}^k x^n \\
        &= \sum_{n=0}^k x^n - \sum_{n=0}^k x^{n+1} \\
        &= x^0 - x^{k+1}
    \end{aligned}
    $$
\end{proof}

\subsection{Some Other Examples}

\begin{example}
    Compute the generating function for the following $a_n$:
    \begin{itemize}
        \item $a_n = 2$ for all $n \geq 0$: $F(x) = \sum_{n=0}^\infty 2 x^n = \frac{2}{1-x}$.
        \item $a_n = 2^n$ for all $n \geq 0$: $F(x) = \sum_{n=0}^\infty 2^n x^n = \sum_{n=0}^\infty (2x)^n = \frac{1}{1-2x}$. 
        \item $a_n = 1$ for $n \geq 5$ and 0 otherwise: There are two methods for this problem.
        
        \textbf{Method 1}: Let $c_n = 1$ for all $n \geq 0$. Let $b_n = 1$ for all $n < 5$ and $b_n = 0$ otherwise. Then, $a_n = c_n - b_n$ for all $n \geq 0$. Then, $F(x) = \sum_{n=0}^\infty (c_n - b_n) x^n = \sum_{n=0}^\infty c^n x^n - \sum_{n=0}^4 b_n x^n = \frac{1}{1-x} - \frac{1-x^5}{1-x}$.

        \textbf{Method 2}: $F(x) = \sum_{n=0}^\infty a_n x^n = \sum_{n=5}^\infty x^n = \sum_{n=0}^\infty x^{n+5} = \frac{x^5}{1-x}$.
        \item $a_n = 1$ for $n \leq 5$ and 0 otherwise: $F(x) = \sum_{n=0}^5 a_n x^n$.
        \item $a_n = 1$ when $n$ is even, and 0 otherwise: $F(x) = \sum_{n=0}^\infty x^{2n} = \sum_{n=0}^\infty (x^2)^n = \frac{1}{1-x^2}$.
        \item $a_n = 2^n + 1$ for all $n \geq 0$: $F(x) = \sum_{n=0}^\infty 2^n x^n + \sum_{n=0}^\infty x^n = \frac{1}{1-2x} + \frac{1}{1-x}$ by linearity.
    \end{itemize}
\end{example}

We can generalize some of the observations while working on the examples above so that we can manipulate generating functions.

\section{Manipulating Generating Functions}

\subsection{Addition}

\begin{lemma}[Linearity of (Ordinary) Generating Functions]
    Let $a_n$ and $b_n$ be two sequences. Let $F(x)$ and $G(x)$ be the ordinary generating functions associated with $a_n$ and $b_n$, respectively. Let $c_n = a_n + b_n$. Then, the generating function associated with $c_n$ is $F(x) + G(x)$.
\end{lemma}

\subsection{Differentiation}

\begin{proof}
    $$
    \begin{aligned}
        F(x) + G(x) &= \sum_{n=0}^\infty a_n x^n + \sum_{n=0}^\infty b_n x^n \\
        &= \sum_{n=0}^\infty (a_n x^n + b_n x^n) & \text{linearity of summation} \\
        &= \sum_{n=0}^\infty (a_n + b_n) x^n \\
        &= \sum_{n=0}^\infty c_n x^n & \text{definition of $c_n$}
    \end{aligned}
    $$
\end{proof}

\begin{theorem}[Derivative of (Ordinary) Generating Function]
    Let $a_n$ be a sequence and $b_{n-1} = na_n$ for $n \geq 1$. If $F(x)$ is the generating function associated with $a_n$, then $F'(x)$ (the first derivative of $F(x)$) is the generating function associated with $b_n$.
\end{theorem}

\begin{proof}
    $$
    \begin{aligned}
        F'(x) &= \sum_{n=0}^\infty a_n n x^{n-1} \\
        &= \sum_{n=1}^\infty a_n n x^{n-1} \\
        &= \sum_{n=1}^\infty b_{n-1} x^{n-1}
    \end{aligned}
    $$
\end{proof}

Differentiation is often used to find the coefficient of the $k$th term given a generating function $F(x)$. To get the coefficient of the $k$th term, we find the $k$th derivative of $F(x)$, $G(x) = \frac{d^k}{dx^k} F(x)$. We note that the coefficient for $x^0$ is $k! \cdot a_k$. We can then substitue $0$ for $x$ and we have $G(x) = k! \cdot a_k$.

\subsection{Multiplication}

\begin{theorem}[Multiplication of (Ordinary) Generating Functions]
    Let $F(x) = \sum_{n=0}^\infty a_n x^n$ and $G(x) = \sum_{n=0}^\infty b_n x^n$. Then,
    $$
    H(x) = F(x) G(x) = \sum_{n=0}^\infty c_n x^n
    $$
    where $c_n = \sum_{k=0}^n a_{n-k} b_k$.
\end{theorem}

\begin{proof}
    Note that
    $$
    \begin{aligned}
        F(x) G(x) &= \sum_{m=0}^\infty a_n G(x) x^m \\
        &= \sum_{l=0}^\infty a_n \sum_{n=0}^\infty b_l x^{m+l}
    \end{aligned}
    $$
    Now we consider the $n$th coefficient of the infinite sum. The term is $x^n$ so for that $n$, $m + l = n$. It follows that the coefficient associated with the term is $\sum_{m+l = n} a_m b_l$. This can be then expressed as $\sum_{k=0}^n a_{n-k} b_{k}$.
\end{proof}

\begin{theorem}
    Let $F_1(x),\ldots,F_m(x)$ be generating functions with $F_i(x) = \sum_{n=0}^\infty a_{n,i} x^n$. Then, 
    $$
    F_1(x) \times \cdots \times F_m(x) = \sum_{n=0}^\infty c_n x^n
    $$
    where
    $$
    c_n = \sum_{e_1+\ldots+e_m = n} a_{e_1,1} + \cdots + a_{e_m,m}.
    $$
\end{theorem}

\begin{proof}
    By induction on $m$.
\end{proof}

\subsection{Coefficient}

If we have an ordinary generating function in its closed form (instead of its expansion), we can get the coefficient of the $k$th term by taking the $k$th derivative. Recall that the coefficient of the $k$th term is also the $k$th element of the sequence $\{a_n\}$ associated with the generating function.

More specifically, if we have a generating function $F(x) = \sum_{n=0}^\infty a_n x^n$ for some sequence $\{a_n\}$, the $k$th coefficient of $F(x)$ or the $k$th element of $\{a_n\}$ is
$$
\frac{\left. \frac{d^k}{dx^k} F(x) \right|_{x=0}}{k!}
$$
That is, the $k$th derivative of $F(x)$ evaluated at 0, divided by $k!$.

This is true because by taking the $k$th derivative, what was originally the $x^k$ term becomes the constant term. And plugging in 0 kills off all terms except the constant term. We need to divide the value by $k!$ because we introduced an additional factor of $k!$ while taking the derivative.

From now on, we will use the notation $[x^k](F(x))$ to refer to the coefficient of the $k$th-power term of the generating function $F(x)$.

\subsection{Composition}

Composition of generating functions also have a combinatorial interpretation.

Let $F(x) = \frac{1}{1-x}$ be the generating function that is the geometric series. If $G(x)$ represents the number of ways to build a certain structure on an $n$-element set, then $G(x)^k$ represents the number of ways to split the $n$-element set into an list of $k$ pieces and build a structure on each piece. As long as the const term ($G(0)$) is 0, then $F(G(x))$ represents all ways to break down an $n$-element set into any number of ordered pieces, \textbf{and} build a structure on each piece.

\subsection{Shifting}

Given a generating function $F(x)$ for the sequence $(a_0,a_1,\ldots)$,
$$
(\underbrace{0,\ldots,0}_{n}, a_0, a_1, \ldots)
$$
has the generating function $x^n F(x)$.

Similarly, $(a_k, a_{k+1},\ldots)$ 
has the generating function
$$
\frac{\sum_{i=0}^{k-1} a_i x^i}{x^n}.
$$

Alternatively stated,
$$
zA(z)=\sum_{n = 1}^\infty a_{n-1}z^n \qquad \text{and} \qquad \frac{A(z)-a_0}{z}=\sum_{n=0}^\infty a_{n+1}z^n
$$

\section{Using Generating Functions in Counting Problems}

Generating functions are useful for solving counting problems, especially object distribution problems that are otherwise hard to model. Ordinary generating functions are useful for counting ways to distribute indistinguishable objects.

This is mainly because when we multiply two generating functions, the coefficients come together in a way that automatically accounts for different ways of making combinations of a given size.

\begin{theorem}
    Let $\mathcal{A}$ and $\mathcal{B}$ be classes of objects and let $A(x)$ and $B(x)$ be their generating functions. Then, the class $\mathcal{C} = \mathcal{A} \times \mathcal{B}$ has the generating function $C(x) = A(x) B(x)$.
\end{theorem}

\begin{proof}
    Let $c_n$ be the number of objects of size $n$ in the Cartesian product $\mathcal{C} = \mathcal{A} \times \mathcal{B}$. These objects $c = (a,b)$ are obtained by picking an object $a \in \mathcal{A}$ of size $k \leq n$ and an object $b \in \mathcal{B}$ of size $n-k$. Thus,
    $$
    c_n = \sum_{k=0}^n a_k b_{n-k}
    $$
    Also note
    $$
    C(x) = A(x) B(x) = \sum_{n=0}^\infty \sum_{k=0}^n a_k b_{n-k} x^n
    $$
    So $C(x) = A(x)B(x)$ is the generating function for $\{c_n\}$.
\end{proof}

Another way to interpret this theorem is: If $a_k$ counts all sets of size $k$ of type $\mathcal{A}$, and $b_k$ counts all sets of size $k$ of type $\mathcal{B}$, then $[x^k](A(x)B(x))$ counts all pairs of sets $(A,B)$ where the total number of elements in both sets is $k$.

Once we have the generating function, we can either find the power series expansion or use derivative to find the coefficient.

\section{Generating Function and Recurrence}

Generating function is also a great tool for solving recurrence relations.

