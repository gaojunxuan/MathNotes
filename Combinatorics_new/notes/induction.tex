\section{Induction}

\begin{definition}
    A mathematical statement or a \textit{\textbf{statement}} is a sentence or expression which can be true or false.
\end{definition}

1 + 1 = 2 is a statement, but 1 + 1 is not a statement. An open statement is a statement that depends on some variables.

\subsection{Principle of Induction}

Note that in this course, $\N = \{1,2,\ldots\}$.

\textbf{Principle of Weak Induction}: Suppose we have a family of statements $P(n)$ for all $n \in \N$, and suppose the following holds:
\begin{enumerate}
    \item $P(1)$ holds, and
    \item if $P(k)$ holds, then $P(k+1)$ holds.
\end{enumerate}
Then, we can conclude $P(n)$ is true for all $n \in \N$.

\begin{remark}
    The base case does not have to be for $n=1$ if we are not proving the statement for all $n \in \N$. Say, for example, that we want to show $P(n)$ holds for all $n \in \N$ such that $n \geq 5$. Then, we can use $P(5)$ as the base case.
\end{remark}

For example, suppose we want to show $\forall n \in \N.\, \sum_{i=0}^n 2^i = 2^{n+1} - 1$.

\begin{proof}
    Let $P(n)$ be the statement
    $$
    ``\sum_{i=0}^{n}2^i = 2^{n+1} - 1"
    $$
    \textbf{1. Base Case}: When $n=1$, LHS = $\sum_{i=0}^n 2^i = 2^0 + 2^1 = 3$. RHS = $2^{i+1} - 1 = 3$. LHS = RHS, so $P(1)$ holds.
    
    \textbf{2. Inductive Case}: Assume $P(k)$ holds (inductive hypothesis). WTS $P(k+1)$ holds. The LHS of $P(k+1)$ is
    $$
    \text{LHS} = \sum_{i=0}^{k+1} 2^i = \underbrace{\sum_{i=0}^k 2^i}_{\substack{\text{LHS of $P(k)$} \\ \text{by IH, equals} \\ \text{RHS of $P(k)$}}} + 2^{k+1} = (2^{k+1} - 1) + 2^{k+1} = 2\cdot 2^{k+1} - 1 = 2^{k+2} - 1
    $$
    which is equal to RHS of $P(k+1)$.

    Since we have (1) and (2), by the Principle of Induction, we can conclude that $P(n)$ holds for all $n \in \N$.
\end{proof}

\subsection{Strong Induction}

\textbf{Principle of Strong Induction}: Suppose we have a family of statements $P(n)$ for all $n \in \N$. Suppose the following are true:
\begin{enumerate}
    \item $P(1)$ is true, and
    \item if $P(m)$ is true for all $1 \leq m \leq k$, then $P(k+1)$ is true.
\end{enumerate}
Then, we can conclude $P(n)$ is true for all $n \in \N$.

Suppose we want to show the Fundamental Theorem of Arithmetics (without the uniqueness part), which states that for all $n \in \N$ such that $n \geq 2$, $n$ is a product of primes.

\begin{proof}
    Let $P(n)$ be that $P(n)$ = ``$n$ is a product of primes''.

    \textbf{1. Base Case}: $P(2)$. Since 2 is a prime, it is a product of primes.

    \textbf{2. Inductive Case}: Suppose that for all $m$ such that $2 \leq m \leq k$, $P(m)$ holds. WTS $P(k+1)$ holds. If $k+1$ is prime, then $P(k+1)$ trivially holds. Otherwise, $k+1$ is composite, so there exists $a,b \in \N$ with $2 \leq a,b < k+1$ such that $k+1 = ab$. So since $P(a)$ and $P(b)$ hold by inductive hypothesis, $a$ and $b$ are both products of primes. So $k+1 = ab$ is a product of primes, so $P(k+1)$ holds.
\end{proof}

\section{Recursion}

As an example, consider the Catalan numbers $C_n = \frac{1}{n+1} \binom{2n}{n}$ for $n \geq 0$. Note that $C_0 = 1$ because of the empty path and $C_1 = 1$. Prove the recursive identity
$$
C_{n+1} = \sum_{i=1}^{n+1} C_{i-1} C_{n+1-i}
$$
for $n \geq 0$.

\begin{proof}
    Let $p = p_1p_2\ldots p_{2n}$ be a Dyck path of length $2n$. Let $p_1\ldots p_{2i}$ be the segment of $p$ up until the first time it touches $x=y$ (but not crosses). So, $p_1=$ E and $p_{2i}=$ N. $p_i$ cannot go North because otherwise we would cross the diagonal. For a similar reason, $p_{2i}$ must go North.

    Then, $p_2\ldots p_{2i-1}$ is a Dyck path of length $2(i-1)$. We can split into cases based on the first time after $(0,0)$ where the path touches the diagonal. It can be at $i = 1,\ldots,n+1$. At a given $i$, we break into two Dyck paths: $p_2 \ldots p_{2i-1}$ of length $2(i-1)$ and $p_{2i+1}\ldots p_{2(n+1)}$ of length $2(n+1-i)$. The first path have $C_{i-1}$ possibilities, and the second path have $C_{n+1-i}$ possibilities.

    In total, we have $\sum_{i=i}^{n+1} C_{i-1} C_{n+1-i}$ number of Dyck paths of length $n+1$.
\end{proof}

\begin{definition}
    Let $\{a_n\}_{n\geq 1}$ be a sequence of real numbers. We say that this sequence satisfies a \textit{\textbf{recurrence relation}} if we can define the $n$th term based on the terms that come before it.
\end{definition}

What we proved just now is a recurrence formula for the Catalan numbers. Namely,
$$
C_{n+1} = \sum_{i=1}^{n+1} C_{i-1} C_{n+1-i}
$$
is a recurrence relation on $\{C_n\}_{n\geq 0}$ with $C_0 = 1$.

\subsection{Proving Recurrences}

Consider the Fibonacci numbers with $F_1 = 1$ and $F_2 = 1$, given by the recurrence relation
$$
F_n = F_{n-1} + F_{n-2} \qquad n \geq 3
$$

Induction is a method of proving statements. Recursion is a way of defining/giving a formula for some quantities.

\begin{example}
    We have a closed form expression of $F_n$. Let $\varphi = \frac{1+\sqrt{5}}{2} \approx 1.6$, and let $\psi = \frac{1-\sqrt{5}}{2}$ be the conjugate of $\varphi$. These are the roots of $x^2 = x + 1$. We would like to show that
    $$
    F_n = \frac{\varphi^{n} - \psi^{n}}{\sqrt{5}}
    $$ 
    for all $n \geq 1$.
\end{example}

We prove this closed form formula using induction.

\begin{proof}
    Let $P(n)$ be the statement we want to prove.

    1. Base Case: $P(1)$. LHS $= F_1 = 1$. RHS $= \frac{\varphi^2 - \psi^2}{\sqrt{5}} = \frac{\left( \frac{1+\sqrt{5}}{2}\right) - \left( \frac{1-\sqrt{5}}{2} \right)}{\sqrt{5}} = \frac{2 \frac{\sqrt{5}}{2}}{\sqrt{5}} = 1$.

    2. Base Case: $P(2)$. LHS = $F_2 = 1$. RHS = $\frac{\varphi^2 - \psi^2}{\sqrt{5}} = \frac{\varphi - \psi}{\sqrt{5}} = 1$.

    3. Inductive Case: Assume $P(n)$ holds for all $1 \leq m \leq k$. WTS $P(k+1)$ holds. Consider the LHS of $P(k+1)$. We have
    $$
    F_{k+1} = F_k + F_{k-1} = \frac{\varphi^k - \psi^k}{\sqrt{5}} + \frac{\varphi^{k-1} - \psi^{k-1}}{\sqrt{5}}
    $$
    by IH. Collect like terms
    $$
    \text{LHS} = \frac{\varphi^{k-1} (\varphi + 1) - \psi^{k-1}(\psi + 1)}{\sqrt{5}} = \frac{\varphi^{k-1}(\varphi^2) - \psi^{k-1}(\psi^2)}{\sqrt{5}}
    $$
    since $\varphi$ and $\psi$ are the solutions to $x^2 = x + 1$. This implies that
    $$
    \text{LHS} = \frac{\varphi^{k+1} - \psi^{k+1}}{\sqrt{5}} = \text{RHS}
    $$
    So $P(k+1)$ holds.
\end{proof}

\begin{remark}
    If instead, we prove the statement
    $$
    P(n) = ``F_n = \frac{\varphi^n - \psi^n}{\sqrt{5}} \land F_{n+1} = \frac{\varphi^{n+1}-\psi^{n+1}}{\sqrt{5}}"
    $$
    We can, in fact, use weak induction.
\end{remark}

Let $a_n$ be the max number of intersections of $n$ lines in the plane. Find a recurrence relation for $a_n$. $a_0 = 0$, $a_1 = 0$, $a_2 = 1$, and $a_3 = 3$. Note that two lines intersect unless parallel. Any two parallel lines can be nudged slightly to not be parallel. Moreover, if 3 lines intersect at a point, we can nudge them so they intersect at distinct pairwise intersections. If we know $a_n$, we can have $a_{n+1} = a_n + n$.